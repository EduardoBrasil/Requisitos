\begin{itemize}
	\item \textbf{Entrevista realizada dia 16 de outubro de 2014}

		\textbf{Cliente}: George Marsicano

		\textbf{Questão 1}: Por que criar uma nova ferramenta de requisitos? As existentes nao te agradram?

		\textbf{Resposta do cliente}:

			Não me agradam? *risos*

			\ \ \ A necessidade de criação ou não de uma ferramente não vem da necessidade de ferramente em si.
			Vocês em primeiro passo estão modelando um processo de requisitos. Depois será realizado uma avaliação para saber se existe ou não alguma ferramente que implemente este processo definido, e, por ultimo, caso não exista nenhuma ferramente, poderá ser desenvolvido um plug-in para alguma existente que seja possivel tal adaptação, ou o desenvolvimento de uma nova ferramente por completo.

			\ \ \ Tudo dependerá a modelagem inicial para decidir o andamento e escopo do projeto.
\end{itemize}