% !TEX root = ../main.tex

\appendix
\chapter{Entrevistas}
	\begin{itemize}
		\item \textbf{Entrevista realizada dia 16 de outubro de 2014}

			\cliente~ George Marsicano\\
			Foi planejado pela equipe de desenvolvimento uma entrevista superficial com seis perguntas, porêm, após a primeira pergunta respondida não foi mais necessário as outras perguntas, salvo a pergunta de número dois.\\
			São elas:\\
			

			\textbf{1. Por que criar uma nova ferramenta de requisitos? As existentes nao te agradram?}
		
			\textbf{R:} Não me agradam?

			A necessidade de criação ou não de uma ferramenta não vem da necessidade de ferramente em si.
			\\
			Vocês em primeiro passo estão modelando um processo de requisitos. Depois será realizado uma avaliação para saber se existe ou não alguma ferramenta que implemente este processo definido, e, por ultimo, caso não exista nenhuma ferramente, poderá ser desenvolvido um plug-in para alguma existente que seja possível tal adaptação, ou o desenvolvimento de uma nova ferramente por completo.
			\\
			Tudo dependerá a modelagem inicial para decidir o andamento e escopo do projeto.
			\\
			\\
			\textbf{2. Se fosse possível resumir a sua necessidade em uma palavra, qual seria?}

			\textbf{R:} Abrangência.
	\end{itemize}

\chapter{Priorização dos casos de uso}

Para a realização da priorização dos Casos de Uso que devem ser implementados utilizou-se uma entrevista com o cliente. A entrevista foi organizada da forma na qual as perguntas seriam apenas os nomes dos Casos de Uso e as respostas poderiam ser \textbf{Alta Prioridade}, \textbf{Média Prioridade} e \textbf{Baixa Prioridade}.

A entrevista e suas respostas estão dispostas na seguinte tabela:

\begin{table}[H]
\centering
\begin{tabular}{|p{7cm}|p{3cm}|}
%----------------------------------------------
\hline
\textbf{Manter Problema:} &
\textcolor{red}{Prioridade Baixa}
\\ \hline
%----------------------------------------------
\textbf{Manter Necessidades:} &
\textcolor{red}{Prioridade Baixa}
\\ \hline
%----------------------------------------------
\textbf{Manter Características:} &
\textcolor{red}{Prioridade Baixa}
\\ \hline
%----------------------------------------------
\textbf{Manter Casos de Uso:} &
\textcolor{red}{Prioridade Baixa}
\\ \hline
%----------------------------------------------
\textbf{Manter Temas de Investimento:} &
\textcolor{red}{Prioridade Baixa}
\\ \hline
%----------------------------------------------
\textbf{Manter Épicos:} &
\textcolor{red}{Prioridade Baixa}
\\ \hline
%----------------------------------------------
\textbf{Manter Features:} &
\textcolor{red}{Prioridade Baixa}
\\ \hline
%----------------------------------------------
\textbf{Manter Histórias de Usuário:} &
\textcolor{red}{Prioridade Baixa}
\\ \hline
%----------------------------------------------
\textbf{Manter Atributos:} &
\textcolor{red}{Prioridade Baixa}
\\ \hline
%----------------------------------------------
\textbf{Manter RoadMaps:} &
\textcolor{blue}{Prioridade Média}
\\ \hline
%----------------------------------------------
\textbf{Manter Rastreabilidade Horizontal entre os requisitos:} &
\textcolor{blue}{Prioridade Média}
\\ \hline
%----------------------------------------------
\textbf{Manter Rastreabilidade Vertical entre os requisitos:} &
\textcolor{blue}{Prioridade Média}
\\ \hline
%----------------------------------------------
\textbf{Definir Metodologia:} &
\textcolor{green}{Prioridade Alta}
\\ \hline
%----------------------------------------------
\textbf{Definir “hibridez” do projeto:} &
\textcolor{green}{Prioridade Alta}
\\ \hline
%----------------------------------------------
\textbf{Manter processos híbridos:} &
\textcolor{green}{Prioridade Alta}
\\ \hline
%----------------------------------------------
\textbf{Gerar Diagrama de Ishikawa:} &
\textcolor{red}{Prioridade Baixa}
\\ \hline
%----------------------------------------------
\textbf{Gerar Diagramas de Casos de Uso:} &
\textcolor{blue}{Prioridade Média}
\\ \hline
%----------------------------------------------
\textbf{Gerar plano de iteração:} &
\textcolor{red}{Prioridade Baixa}
\\ \hline
%----------------------------------------------
\textbf{Controlar histórico de versão:} &
\textcolor{red}{Prioridade Baixa}
\\ \hline
%----------------------------------------------
\end{tabular}
\caption{Entrevista Priorização de Casos de Uso}
\label{tab:frameworkproblema}
\end{table}
