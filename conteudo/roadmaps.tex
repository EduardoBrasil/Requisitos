% !TEX root = ../main.tex

\textit{Roadmaps} são uma priorização dos recursos do sistema, para definir por qual requisito a implementação terá início.

Para gerar o \textit{roadmap} foi gerada uma pontuação nos atributos dos recursos presentes nas tabelas \ref{tab:atributo_arquitetura} e \ref{tab:atributo_prioridade}, mostrada na tabela \ref{tab:pontuacao_atributos}.

\begin{table}[H]
\centering
\begin{tabular}{|p{2cm}|p{5cm}|p{3cm}|}

\hline
\textbf{Atributo} &
\textbf{Classificação} &
\textbf{Pontuação}
\\ \hline

%------------------------------------------
\multirow{3}{*}{
\textbf{Prioridade}} &
	Alta prioridade &
	5
	\\ \cline{2-3} &
	Média prioridade  &
	3
	\\ \cline{2-3} &
	Baixa prioridade  &
	1
	\\ \hline

%------------------------------------------
\multirow{4}{*}{\textbf{Arquitetura}} &
	Grande &
	7
	\\ \cline{2-3} &
	Média &
	5
	\\ \cline{2-3} &
	Baixa &
	3
	\\ \cline{2-3} &
	Nenhuma &
	1
	\\ \hline
%--------------------------------------------
\end{tabular}
\caption{Pontuação dos Atributos}
\label{tab:pontuacao_atributos}
\end{table}

Utilizando a tabela \ref{tab:pontuacao_atributos}, fomos capazes de fazer uma relação numérica para pontuar cada um dos recursos, apresentados na sessão \ref{subsub:recursos_produto} deste documento, e fazer a escolha de qual deve ser implementado primeiro, esta relação está apresentada na tabela \ref{tab:pontuacao_recursos}.

\begin{table}[h]
\centering
\begin{tabular}{|l|C{3cm}|C{3cm}|l|}

\hline
\textbf{Recurso} &
\textbf{Atributo de prioridade} &
\small{\textbf{Atributo de arquitetura}} &
\textbf{Pontuação final}
\\ \hline

%---------------------------------------------
Definir Metodologia & 
Alta prioridade &
Média &
10
 \\ \hline

 %--------------------------------------------
Manter Problema &
Alta prioridade &
Grande &
12
 \\ \hline
 
 %--------------------------------------------
Manter Necessidades &
Alta prioridade &
Grande &
12
 \\ \hline
 
 %--------------------------------------------
Manter Características &
Alta prioridade &
Grande &
12
 \\ \hline
 
 %--------------------------------------------
Manter Casos de Uso &
Alta prioridade &
Grande &
12
 \\ \hline
 
 %--------------------------------------------
Manter Temas de Investimento &
Alta prioridade &
Grande &
12
 \\ \hline
 
 %--------------------------------------------
Manter Épicos &
Alta prioridade &
Grande &
12
 \\ \hline
 
 %--------------------------------------------
Manter Features &
Alta prioridade &
Grande &
12
 \\ \hline
 
 %--------------------------------------------
Manter Histórias de Usuário &
Alta prioridade &
Grande &
12
 \\ \hline
 
 %--------------------------------------------
Manter Atributos &
Média prioridade &
Baixa &
6
 \\ \hline
 
 %--------------------------------------------
Manter Rastreabilidade de Requisitos &
Alta prioridade &
Média &
10
 \\ \hline
 
 %--------------------------------------------
Gerar \textit{Diagrama de Ishikawa} &
Média prioridade &
Média &
8
 \\ \hline
 
 %--------------------------------------------
Manter Atores do Projeto &
Baixa prioridade &
Baixa &
4
 \\ \hline
 
 %--------------------------------------------
Gerar Diagramas de Casos de Uso &
Média prioridade &
Média &
8
 \\ \hline
 
 %--------------------------------------------
Manter \textit{Roadmaps} &
Média prioridade &
Média &
8
 \\ \hline
 
 %--------------------------------------------
Gerar plano de iteração &
média prioridade &
Nenhuma &
4
 \\ \hline
 
 %--------------------------------------------
Definir ``hibridez'' do projeto &
Alta prioridade &
Grande &
12
 \\ \hline

 %--------------------------------------------
Controlar histórico de versão &
Baixa prioridade &
Nenhuma &
2
 \\ \hline
 %--------------------------------------------
\end{tabular}
\caption{Pontuação dos recursos}
\label{tab:pontuacao_recursos}
\end{table}

Utilizando os dados apresentados na tabela \ref{tab:pontuacao_recursos}, podemos então gerar um \textit{rank} da ordem em que as funcionalidades devem ser implementadas, e a ordem deve ser de acordo com a lista a baixo:

\begin{enumerate}
	\item \textbf{Primeira prioridade:}
		\begin{itemize}
			\item Manter problema;
			\item Manter Necessidade;
			\item Manter Características;
			\item Manter Casos de uso;
			\item Manter Temas de investimento;
			\item Manter Épicos;
			\item Manter Features;
			\item Manter Histórias de usuário;
			\item definir ``hibridez'' do projeto.
		\end{itemize}
	\item \textbf{Segunda prioridade:}
		\begin{itemize}
			\item Definir metodologia;
			\item Manter rastreabilidade de requisitos;
		\end{itemize}
	\item \textbf{Terceira prioridade:}
		\begin{itemize}
			\item Gerar \textit{Diagrama de Ishikawa};
			\item Gerar Diagrama de Caso de Uso;
			\item Manter \textit{roadmaps}.
		\end{itemize}
	\item \textbf{Quarta prioridade:}
		\begin{itemize}
			\item Manter Atributos.
		\end{itemize}
	\item \textbf{Quinta prioridade:}
		\begin{itemize}
			\item Manter atores do projeto;
			\item gerar planos de iteração.
		\end{itemize}
	\item \textbf{Sexta prioridade:}
		\begin{itemize}
			\item Controlar histórico de versão.
		\end{itemize}
\end{enumerate}

Após estudo e análise das metodologias possíveis para desenvilvimento da ferramenta, escolheu-se a utilização da metodologia Ágil, graças ao pequeno tempo para desenvolvimento, a disponibilidade do cliente e o tamanho da equipe. Como o tempo de desenvolvimento será bastante curto, não serão implementados todos os requisitos do sistema. Dessa forma, apresenta-se o \textit{roadmap} utilizado nas iterações que serão realizados na tabela \ref{tab:primeiro_roadmap}.

\vspace{5mm}
\begin{table}[h]
\centering
\begin{tabular}{p{1cm}|p{6cm}|p{5,5cm}|}

\cline{2-3} &
\textbf{Iteração 1} &
\textbf{Iteração 2}
\\ \hline
%-----------------------------------------
\multicolumn{1}{|p{1cm}|}{\textbf{Casos de uso}} &
\begin{itemize}
 	\item Definir metodologia;
 	\item Manter Temas de investimento;
	\item Manter Épicos.
\end{itemize} &
\begin{itemize}
 	\item Manter Features;
	\item Manter Histórias de usuário.
 \end{itemize} 
 \\ \hline
\end{tabular}
\caption{\textit{Roadmap}}
\label{tab:primeiro_roadmap}
\end{table}

Este \textit{roadmap} será utilizado mais a frente, na sessão de \ref{sec:documento_de_caso_de_uso} para auxiliar quais casos de uso devem ou não ser detalhados durante o processo de desenvolvimento.