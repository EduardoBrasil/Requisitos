% !TEX root = ../main.tex

\textit{Roadmaps} são uma priorização dos recursos do sistema, para definir por qual requisito a implementação terá início.

Para gerar o \textit{roadmap} foi gerada uma pontuação nos atributos dos recursos presentes na Tabelas \ref{tab:pontuacao_recursos}, mostrada na Tabela \ref{tab:pontuacao_atributos}.

\begin{table}[H]
\centering
\begin{tabular}{|p{2cm}|p{5cm}|p{3cm}|}

\hline
\textbf{Atributo} &
\textbf{Classificação} &
\textbf{Pontuação}
\\ \hline

%------------------------------------------
\multirow{3}{*}{
\textbf{Prioridade}} &
	Alta prioridade &
	5
	\\ \cline{2-3} &
	Média prioridade  &
	3
	\\ \cline{2-3} &
	Baixa prioridade  &
	1
	\\ \hline

%------------------------------------------
\multirow{4}{*}{\textbf{Arquitetura}} &
	Grande &
	7
	\\ \cline{2-3} &
	Média &
	5
	\\ \cline{2-3} &
	Baixa &
	3
	\\ \cline{2-3} &
	Nenhuma &
	1
	\\ \hline
%--------------------------------------------
\end{tabular}
\caption{Pontuação dos Atributos}
\label{tab:pontuacao_atributos}
\end{table}

Utilizando a Tabela \ref{tab:pontuacao_atributos}, fomos capazes de fazer uma relação numérica para pontuar cada um dos recursos, apresentados na Sessão \ref{subsub:recursos_produto} deste documento, e fazer a escolha de qual deve ser implementado primeiro, esta relação está apresentada na Tabela \ref{tab:pontuacao_recursos}.

\begin{table}[H]
\centering
\begin{tabular}{|C{5cm}|C{3cm}|C{3cm}|C{3cm}|}

\hline
\textbf{Recurso} &
\textbf{Atributo de prioridade} &
\small{\textbf{Atributo de arquitetura}} &
\textbf{Pontuação final}
\\ \hline

%---------------------------------------------
Definir Metodologia & 
\altaPrioridade &
\altoRisco &
12
 \\ \hline
 %--------------------------------------------
Definir ``hibridez'' do projeto &
\altaPrioridade &
\altoRisco &
12
 \\ \hline

 %--------------------------------------------
Manter processos híbridos &
\altaPrioridade &
\medioRisco &
10
 \\ \hline

 %--------------------------------------------
Gerar Diagramas de Casos de Uso &
\mediaPrioridade &
\medioRisco &
8
 \\ \hline
 
 %--------------------------------------------
Manter \textit{Roadmaps} &
\mediaPrioridade &
\medioRisco &
8
 \\ \hline
 %--------------------------------------------
Manter Rastreabilidade vertical entre os Requisitos &
\mediaPrioridade &
\baixoRisco &
6
 \\ \hline
 %--------------------------------------------
Gerar \textit{Diagrama de Ishikawa} &
\baixaPrioridade &
\medioRisco &
6
 \\ \hline
 
 
 %--------------------------------------------
Manter Rastreabilidade Horizontal entre os Requisitos &
\mediaPrioridade &
\baixoRisco &
6
 \\ \hline
 %--------------------------------------------
Controlar histórico de versão &
\baixaPrioridade &
\medioRisco &
6
 \\ \hline

 %--------------------------------------------
Manter Problema &
\baixaPrioridade &
\baixoRisco &
4
 \\ \hline
 
 %--------------------------------------------
Manter Necessidades &
\baixaPrioridade &
\baixoRisco &
4
 \\ \hline
 
 %--------------------------------------------
Manter Características &
\baixaPrioridade &
\baixoRisco &
4
 \\ \hline
 
 %--------------------------------------------
Manter Casos de Uso &
\baixaPrioridade &
\baixoRisco &
4
 \\ \hline
 
 %--------------------------------------------
Manter Temas de Investimento &
\baixaPrioridade &
\baixoRisco &
4
 \\ \hline
 
 %--------------------------------------------
Manter Épicos &
\baixaPrioridade &
\baixoRisco &
4
 \\ \hline
 
 %--------------------------------------------
Manter Features &
\baixaPrioridade &
\baixoRisco &
4
 \\ \hline
 
 %--------------------------------------------
Manter Histórias de Usuário &
\baixaPrioridade &
\baixoRisco &
4
 \\ \hline
 
 %--------------------------------------------
Manter Atores do Projeto &
\baixaPrioridade &
\baixoRisco &
4
 \\ \hline
 
Manter Atributos &
\baixaPrioridade &
\nenhumRisco &
2
 \\ \hline
 %--------------------------------------------
 
 %--------------------------------------------
Gerar plano de iteração &
\baixaPrioridade &
\nenhumRisco &
2
 \\ \hline
 


\end{tabular}
\caption{Pontuação dos recursos}
\label{tab:pontuacao_recursos}
\end{table}

Utilizando os dados apresentados na Tabela \ref{tab:pontuacao_recursos}, podemos então gerar um \textit{ranking} da ordem em que as funcionalidades devem ser implementadas, e a ordem deve ser de acordo com a lista a baixo:

\begin{enumerate}
	\item \textbf{Primeira prioridade:}
		\begin{itemize}
			\item Definir metodologia;
			\item Definir ``hibridez'' do projeto.
		\end{itemize}
	\item \textbf{Segunda prioridade:}
		\begin{itemize}
			\item Manter processos híbridos;
		\end{itemize}
	\item \textbf{Terceira prioridade:}
		\begin{itemize}
			\item Gerar diagrama de caso de uso;
			\item Manter \textit{roadmaps}.
		\end{itemize}
	\item \textbf{Quarta prioridade:}
		\begin{itemize}
			\item Gerar \textit{Diagrama de Ishikawa};
			\item Manter rastreabilidade horizontal entre os requisitos;
			\item Manter rastreabilidade vertical entre os requisitos;
			\item Controlar histórico de versão;
		\end{itemize}
	\item \textbf{Quinta prioridade:}
		\begin{itemize}
			\item Manter problema;
			\item Manter necessidade;
			\item Manter características;
			\item Manter casos de uso;
			\item Manter temas de investimento;
			\item Manter épicos;
			\item Manter deatures;
			\item Manter histórias de usuário;
		\end{itemize}
	\item \textbf{Sexta prioridade:}
		\begin{itemize}
			\item Manter atributos;
			\item Gerar plano de iteração;
		\end{itemize}
\end{enumerate}

Utilizando este ranking foi possível gerar um \textit{roadmap} baseado nas prioridades do cliente, e na complexidade do código, este \textit{roadmap} está representado na Tabela \ref{tab:primeiro_roadmap}  

\vspace{5mm}
\begin{table}[H]
\centering
\begin{tabular}{p{1cm}|p{6cm}|p{5,5cm}|}

\cline{2-3} &
\textbf{Iteração 1} &
\textbf{Iteração 2}
\\ \hline
%-----------------------------------------
\multicolumn{1}{|p{1cm}|}{\textbf{Casos de uso}} &
\begin{itemize}
 	\item Definir metodologia;
\end{itemize} &
\begin{itemize}
	\item Definir ``hibridez'' do projeto.
 	\item Manter processos híbridos.
 \end{itemize} 
 \\ \hline
\end{tabular}
\caption{\textit{Roadmap}}
\label{tab:primeiro_roadmap}
\end{table}

Este \textit{roadmap} será utilizado mais a frente, na Sessão de \ref{sec:documento_de_caso_de_uso} para auxiliar quais casos de uso devem ou não ser detalhados durante o processo de desenvolvimento.