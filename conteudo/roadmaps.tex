% !TEX root = ../main.tex

\textit{Roadmaps} são uma priorização dos recursos do sistema, para definir por qual requisito a implementação terá início.

Para gerar o \textit{roadmap} foi gerada uma pontuação nos atributos dos recursos presentes nas tabelas \ref{tab:atributo_arquitetura} e \ref{tab:atributo_prioridade}, mostrada na tabela \ref{tab:pontuacao_atributos}.

\begin{table}[H]
\centering
\begin{tabular}{|p{2cm}|p{5cm}|p{3cm}|}

\hline
\textbf{Tipo do Atributo} &
\textbf{Atributo} &
\textbf{Pontuação}
\\ \hline

%------------------------------------------
\multirow{3}{*}{
\textbf{Prioridade}} &
	Alta prioridade &
	5
	\\ \cline{2-3} &
	Média prioridade  &
	3
	\\ \cline{2-3} &
	Baixa prioridade  &
	1
	\\ \hline

%------------------------------------------
\multirow{4}{*}{\textbf{Arquitetura}} &
	Grande &
	7
	\\ \cline{2-3} &
	Média &
	5
	\\ \cline{2-3} &
	Baixa &
	3
	\\ \cline{2-3} &
	Nenhuma &
	1
	\\ \hline
%--------------------------------------------
\end{tabular}
\caption{Pontuação dos Atributos}
\label{tab:pontuacao_atributos}
\end{table}

Utilizando a tabela \ref{tab:pontuacao_atributos}, fomos capazes de fazer uma relação numérica para pontuar cada um dos recursos, apresentados na sessão \ref{subsub:recursos_produto} deste documento, e fazer a escolha de qual deve ser implementado primeiro, esta relação está apresentada na tabela \ref{tab:pontuacao_recursos}.

\begin{table}[h]
\centering
\begin{tabular}{|p{3.5cm}|p{4cm}|p{4.5cm}|p{2cm}|}

\hline
\textbf{Recurso} &
\textbf{Atributo de prioridade} &
\textbf{Atributo de arquitetura} &
\textbf{Pontuação final}
\\ \hline

%---------------------------------------------
Definir Metodologia & 
 &
 &
 \\ \hline

 %--------------------------------------------
Manter Problema &
 &
 &
 \\ \hline
 
 %--------------------------------------------
Manter Necessidades &
 &
 &
 \\ \hline
 
 %--------------------------------------------
Manter Características &
 &
 &
 \\ \hline
 
 %--------------------------------------------
Manter Casos de Uso &
 &
 &
 \\ \hline
 
 %--------------------------------------------
Manter Temas de Investimento &
 &
 &
 \\ \hline
 
 %--------------------------------------------
Manter Épicos &
 &
 &
 \\ \hline
 
 %--------------------------------------------
Manter Features &
 &
 &
 \\ \hline
 
 %--------------------------------------------
Manter Histórias de Usuário &
 &
 &
 \\ \hline
 
 %--------------------------------------------
Manter Atributos &
 &
 &
 \\ \hline
 
 %--------------------------------------------
Manter Rastreabilidade de Requisitos &
 &
 &
 \\ \hline
 
 %--------------------------------------------
Gerar \textit{Diagrama de Ishikawa} &
 &
 &
 \\ \hline
 
 %--------------------------------------------
Manter Atores do Projeto &
 &
 &
 \\ \hline
 
 %--------------------------------------------
Gerar Diagramas de Casos de Uso &
 &
 &
 \\ \hline
 
 %--------------------------------------------
Manter \textit{Roadmaps} &
 &
 &
 \\ \hline
 
 %--------------------------------------------
Gerar plano de iteração &
 &
 &
 \\ \hline
 
 %--------------------------------------------
Definir ``hibridez'' do projeto &
 &
 &
 \\ \hline

 %--------------------------------------------
Controlar histórico de versão &
 &
 &
 \\ \hline
 %--------------------------------------------
\end{tabular}
\caption{Pontuação dos recursos}
\label{tab:pontuacao_recursos}
\end{table}