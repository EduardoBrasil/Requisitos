As especificações suplementares listam todas as definições dos sistema que não estão incluídas no modelo de Caso de uso, ou seja, não estão relacionadas com a funcionalidade em sí do sistema, \ref{rup}.

\subsection{Usabilidade}
	
	A ferramenta \nomeFerramenta~ deve ser de fácil utilização, a ponto de não ser necessário treinamento do usuário para a utilização das funcionalidades básicas, e deve ser intuitiva, para que após um mês, no máximo, o usuário alcance uma produtividade boa em manipular os requisitos e utilizar o sistema.

\subsection{Confiabilidade}

	A ferramenta \nomeFerramenta~ deve estar presente em pelo menos 95\% do tempo, desonsiderando erros do servidor no qual será instalada pelo usuário, e deve ser possuir resiliência suficiente para que não seja necessário reinicialização do sistema a cada quebra.

	Infelizmente, não é possível gerar um sistema 100\% \textit{anti-quebra}, e sabemos que a ferramenta irá passar por falhas. Porém estabelecemos que o tempo mínimo entre falhas do sistema deverá ser de 1 mês.

	O sistema também deverá ser confiável em relação à invasões, devendo ser capaz de impedir os ataques básicos ao sistema, e ao seu banco de dados.

\subsection{Desempenho}

	A ferramenta será considerada com um desempenho bom quando qualquer uma de suas páginas não levar mais de dois segundos para ser carregada, e em média levar apenas um segundo.

	Além do tempo de resposta das páginas outro quesito para a ferramenta possuir um bom desempenho é o número de consultas no banco de dados por funcionalidade, não devendo ser maior que uma.

	O sistema deverá ser capaz também de receber um total de mil usuários simultâneos ter influência na regra do tempo de resposta de cada uma das páginas.