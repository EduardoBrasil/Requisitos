As especificações suplementares listam todas as definições dos sistema que não estão incluídas no modelo de Caso de uso, ou seja, não estão relacionadas com a funcionalidade em sí do sistema, \cite{rup}.

\subsection{Características do sistema}
\label{subSection:suplementares_Caract_sistema}
As características do sistema serão representadas nesta sessão e comtemplam todas as definições que serão usadas para a definição dos requisitos não funcionais.

\subsubsection{Usabilidade}
	
	A ferramenta \nomeferramenta~ deve ser de fácil utilização, a ponto de não ser necessário treinamento do usuário para a utilização das funcionalidades básicas, e deve ser intuitiva, para que após um mês, no máximo, o usuário alcance uma produtividade boa em manipular os requisitos e utilizar o sistema.

\subsubsection{Confiabilidade}

	A ferramenta \nomeferramenta~ deve estar disponível pelo menos 95\% do tempo, desconsiderando erros do servidor no qual será instalada pelo usuário, e deve ser possuir resiliência suficiente para que não seja necessário reinicialização do sistema a cada quebra.

	Infelizmente, não é possível gerar um sistema 100\% \textit{anti-quebra}, e sabemos que a ferramenta irá passar por falhas. Porém estabelecemos que o tempo mínimo entre falhas do sistema deverá ser de 1 mês.

	O sistema também deverá ser confiável em relação à invasões, devendo ser capaz de impedir os ataques básicos ao sistema, e ao seu banco de dados.

\subsubsection{Desempenho}

	A ferramenta será considerada com um desempenho bom quando qualquer uma de suas páginas não levar mais de dois segundos para ser carregada, e em média levar apenas um segundo.

	Além do tempo de resposta das páginas outro quesito para a ferramenta possuir um bom desempenho é o número de consultas no banco de dados por funcionalidade, não devendo ser maior que uma.

	O sistema deverá ser capaz também de receber um total de mil usuários simultâneos ter influência na regra do tempo de resposta de cada uma das páginas.

\subsection{Requisitos não funcionais}

	Os requisitos não funcionais partem das características do sistema, citados na Sessão \ref{subSection:suplementares_Caract_sistema}, e são listados abaixo. 

	\begin{enumerate}
		\item \textbf{Usabilidade:}
			\begin{itemize}
				\item \textbf{RNF01} - O usuário não deve necessitar de treinamento para utilizar a ferramenta;
				\item \textbf{RNF02} - O usuário não deve levar mais de um mês para estar totalmente produtivo na utilização da ferramenta.
			\end{itemize}
		\item \textbf{Confiabilidade:}
			\begin{itemize}
				\item \textbf{RNF03} - O sistema deve estar disponível 95\% do tempo;
				\item \textbf{RNF04} - O sistema deve ser resiliente;
				\item \textbf{RNF05} - O sistema deve ter um espaço mínimo de 1 mês entre uma falha e outra;
				\item \textbf{RNF06} - O sistema deve ser capaz de resistir à invasões conhecidas.
			\end{itemize}
		\item \textbf{Desempenho:}
			\begin{itemize}
				\item \textbf{RNF07} - O sistema deve ter um tempo de resposta em qualquer página de no máximo dois segundos e em média um segundo;
				\item \textbf{RNF08} - O sistema deve ter no máximo 1 consulta ao banco de dados por funcionalidade;
				\item \textbf{RNF09} - O sistema deve suportar até mil usuários sem redução no tempo de resposta das páginas.
			\end{itemize}
	\end{enumerate}