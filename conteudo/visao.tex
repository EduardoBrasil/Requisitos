% !TEX root = ../main.tex

O documento de visão tem como objetivo definir uma visão geral do projeto, apresentar os problemas, os requisitos funcionais, não funcionanis, atores, entre outras informações que serão definidas com o cliente a fim de garantir que a equipe de desenvolvimento e o cliente estejam na maior sincronia possível \cite{IBM:2014:Online}.

%-----------------------------------------------------------------------------------------------------------
\subsection{Posicionando}
\subsubsection{Oportunidade de Negócios}

Atualmente, as ferramentas no mercado possuem limitações, como de qualidade, falta de flexibilidade na gerência, ou até mesmo o fechamento do código, que pode ser considerado uma limitação devida a redução de mão de obra para manutenção e evolução.

\subsubsection{Instrução do Problema}

A \er{} possui diversas metodologias possíveis para se seguir, como, por exemplo as metodologias ágeis, tradicionais ou até mesmo uma mistura das duas.

Infelizmente, cada ferramenta de gerência de requisitos é voltada para uma dessas possibilidades, tornando díficil a tarefa voltada para outras, gerando assim nos engenheiros de requisitos a necessidade de aprender a utilizar diversas ferramentas para gerir projetos com metodologias diferentes.

A utilização de apenas uma ferramenta que abrangesse as duas metodologias e ainda uma mistura das duas resolveria todo problema de gerência de requisitos em projetos que não se adequam perfeitamente a uma metodologia específica.

\subsubsection{Instrução de Posição do Produto}

Para os engenheiros de \er, a \nomeferramenta~ representará um avanço nas atividades de gerenciamento, pois apenas precisarão aprender as funcionalidades de uma ferramenta, simplificando a mudança entre projetos que tomam metodologias distintas.

%-----------------------------------------------------------------------------------------------------------

%-----------------------------------------------------------------------------------------------------------
\subsection{Descrições da Parte Interessada e do Usuário}

Nesta Seção serão identificados e detalhados os interessados e usuários da \nomeferramenta{}.

\subsubsection{Resumo da Parte Interessada e do Usuário}

Para melhor entendimento das características e responsabilidades dos interessados, utilizou-se uma tabela que apresenta todos os interessados no sistema, suas descrições, responsabilidades e os critérios de sucesso de suas funções na equipe, ilustrada na Tabela \ref{tab:parteInteressada}. Com esta tabela, pode-se obter o entendimento necessário sobre os interessados e o quão importante eles são para o sucesso do sistema.

\begin{table}[H]
\centering
\begin{tabular}{|p{2cm}|p{5cm}|p{4cm}|p{4cm}|}
\hline
%-------------------------------------------------------
\textbf{Interessado} &
\textbf{Descrição} &
\textbf{Responsabilidade} &
\textbf{Critérios de Sucesso}
\\ \hline

%-------------------------------------------------------
Analista de Requisitos &
Membro da equipe de desenvolvimento com facilidade em comunicação, psicologia, sociologia, filosofia e mais áreas que possam facilitar a relação com o cliente. Seu conhecimento na área pode ser, dependendo da organização, baixo. &
Pessoa responsável por realizar a elicitação dos requisitos junto ao usuário. Deve elicitar os requisitos de forma adequada à garantir sucesso no desenvolvimento do \sw. &
Requisitos corretamente elicitados e prontos para serem documentados. 
\\ \hline
%-------------------------------------------------------
Gerente de Requisitos &
Conhecedor de todo o processo de desenvolvimento e com contato frequente com o cliente. Seu conhecimento deve ser alto. &
Pessoa responsável por administrar os requisitos durante todo processo de desenvolvimento de \sw, garantindo o mínimo esforço em casos de mudança de requisitos. &
Requisitos bem administrados para, no caso de mudanças nos requisitos, existir o menor impacto possível na equipe de desenvolvimento.
\\ \hline
%--------------------------------------------------------
Programador &
Pessoa com capacidade em linguagens e lógica de programação &
Implementar o sistema utilizando as técnologias definidas &
Implementação do sistema de acordo com os requisitos levantados e cadastrados na ferramenta
\\ \hline

\end{tabular}
\label{}
\caption{Parte Interessada}
\label{tab:parteInteressada}
\end{table}

\subsubsection{Principais Problemas e Necessidades da Parte Interessada}

O problema a ser resolvido pela \nomeferramenta~ deve estar bastante claro entre todos os \stakeholder, para que o desenvolvimento passe pela menor quantidade possível de dificuldades quanto ao entendimento de onde focar esforços para desenvolver a solução.

Para o mapeamento do problema principal e suas causas, foi utilizada a técnica do \textit{Diagrama de Ishikawa}, que se encontra na Figura \ref{img:fishbone}.

%-------------------------------------FISHBONE AQUI----------------------------------------
\begin{figure}[H]
	\centering
	\includegraphics[width=0.8\textwidth]{imgModelagem/fishbone}
	\caption{Diagrama de Ishikawa}
	\label{img:fishbone}
\end{figure}
%-------------------------------------FISHBONE AQUI----------------------------------------


%-------------------------------------FRAMEWORK DE Problema-----------------------------------
Para melhor entendimento do problema, utilizamos a técnica de \textit{Framework de problema}, que consiste em criar uma tabela apresentando o problema, os afetados, o impacto e qual seria uma solução bem sucedida, o Framework está retratado na Tabela \ref{tab:frameworkproblema}.

\begin{table}[H]
\centering
\begin{tabular}{|p{3cm}|p{10cm}|p{2.5cm}|}
%----------------------------------------------
\hline
\textbf{O Problema:} &
Falta de flexibilidade entre Abordagens e Ferramentas. 
\\ \hline
%----------------------------------------------
\textbf{Afeta:} &
Todos os desenvolvedores de \sw~que necessitam de uma flexibilidade maior na gerência de requisitos.
\\ \hline
%----------------------------------------------
\textbf{Cujo impacto é:} &
Processo de requisitos mal gerenciados, aumentando a possibilidade de erros durante o desenvolvimento.
\\ \hline
%----------------------------------------------
\textbf{Uma solução bem sucedida seria:} &
Utilização de uma ferramenta que faça gerência de requisitos de forma flexivel, podendo utilizá-la em qualquer metodologia.
\\ \hline
%----------------------------------------------
\end{tabular}
\caption{Framework de Problema}
\label{tab:frameworkproblema}
\end{table}

%-----------------------------------------FRAMEWORK DE NECESSIDADES----------------------------------------
Após o entendimento do problema, vê-se necessária a documentação das necessidades do cliente. Utilizou-se uma técnica chamada \textit{framework de necessidades} na qual são apresentados todos os problemas, as necessidades, a solução atual e a solução proposta. Dessa forma, pode-se obter um entendimento mais organizado dos problemas e necessidades do cliente, de acordo com o retratado na Tabela \ref{tab:frameworknecessidade}.

\begin{table}[H]
\centering
\begin{tabular}{|p{5cm}|p{3cm}|p{3cm}|p{5cm}|}

%-------------------------------------------------------------
\hline
\textbf{Necessidade} &
\textbf{Problema} &
\textbf{Solução Atual} &
\textbf{Solução Proposta}
\\ \hline

%-------------------------------------------------------------
Utilização de ferramentas que se adequem as metodologias. &
Ferramentas associadas a metodologias específicas. &
Equipe utiliza mais de uma ferramenta para abranger as abordagens utilizadas. &
Criação de uma ferramenta que seja flexível para qualquer metodologia, abrangendo todas as agordagens e até a mesclagem das mesmas.
\\ \hline
%-------------------------------------------------------------
Apoio a utilização de uma rastreabilidade organizada e eficiente em qualquer abordagem. &
Limitação de ratreabilidade. &
A equipe precisa criar sua rastreabilidade sem o apoio de uma ferramenta flexível. &
Criação de uma ferramenta que gere a rastreabilidade dos requisitos de forma organizada e eficiente para qualquer abordagem.
\\ \hline
%-------------------------------------------------------------
Obter critérios fixos que redirecionem o projeto para a abordagem mais adequada. &
Falta de critérios para seleção de abordagens. &
Equipe precisa estudar as características do projeto e decidir qual a abordagem mais adequada. &
Criação de uma ferramenta que recolha as características do projeto e apresente a abordagem mais adequada.
\\ \hline
%-------------------------------------------------------------
Obter um processo de \textit{E.R.} adaptável a qualquer abordagem. &
Falta de definição de um processo de \textit{E.R.} adaptável. &
Utilização de um processo inflexível e voltado apenas para uma abordagem. &
Criação de uma ferramenta que gerencie processos flexíveis.
\\ \hline
%-------------------------------------------------------------
Gerar documentação de qualidade e fácil entendimento &
Dificuldade em gerar documentação para pessoas de fora da equipe &
Utilização de ferramentas a parte para gerar Diagramas de Caso de Uso e \textit{Diagramas de Ishikawa}. &
Integração da documentação na ferramenta de requisitos.
\\ \hline
%-------------------------------------------------------------
\end{tabular}
\caption{Framework de Necessidades}
\label{tab:frameworknecessidade}
\end{table}

%-----------------------------------------------------------------------------------------------------------
\subsection{Visão Geral do Produto}
	
Nesta seção, pode-se ter um entendimento geral de como será o produto final, quais serão suas características, como serão suas funcionalidades e etc.

\subsubsection{Perspectiva do Produto}
	
O produto se encontrará em um contexto onde existem inúmeras ferramentas com o mesmo propósito, porém, as ferramentas existentes são inflexíveis quando se trata da abordagem que será seguida durante o desenvolvimento de \sw. Esta falha será corrigida na \nomeferramenta, que irá propor uma metodologia para cada projeto em particular de acordo com suas características.

A ferramenta pode ser autocontida, não necessitando do apoio de nenhum outro sistema, porém a utilização de ferramentas de modelagem de processos é bastante indicada para que a máxima organização do projeto seja alcançada.

\subsubsection{Resumo das Capacidades}
	
O grande diferencial da \nomeferramenta~ será a flexibilização na abordagem que será seguida durante o gerenciamento de projetos de \sw. O sistema deverá indicar a melhor abordagem a ser seguida pela equipe de desenvolvimento, garantindo a otimização do processo de desenvolvimento.

A ferramenta será capaz de disponibilizar a opção de modificar a abordagem indicada pela ferramenta, para que a equipe de desenvolvimento possa escolher a abordagem na qual os mesmos se sentem mais a vontade.

%-----------------------------------------------------------------------------------------------------------
\subsection{Requisitos Funcionais}
	Requisitos funcionais são as características do sistema associadas às funcionalidades do sistem em sí, como por exemplo o que o sistema deve fazer e como deve se comportar. Os requisitos funcionais estão listados abaixo.

	\begin{enumerate}
		\item \textbf{Característica C1.1.1 - Manter metodologias tradicionais}
			\begin{itemize}
				\item \textbf{RF001} - Cadastrar problema na ferramenta;
				\item \textbf{RF002} - Ler problema na ferramenta;
				\item \textbf{RF003} - Atualizar problema na ferramenta;
				\item \textbf{RF004} - Remover problema na ferramenta;
				\item \textbf{RF005} - Cadastrar necessidade na ferramenta;
				\item \textbf{RF006} - Ler necessidade na ferramenta;
				\item \textbf{RF007} - Atualizar necessidade na ferramenta;
				\item \textbf{RF008} - Remover necessidade na ferramenta;
				\item \textbf{RF009} - Cadastrar características na ferramenta;
				\item \textbf{RF010} - Ler características na ferramenta;
				\item \textbf{RF011} - Atualizar características na ferramenta;
				\item \textbf{RF012} - Remover características na ferramenta;
				\item \textbf{RF013} - Cadastrar casos de uso na ferramenta;
				\item \textbf{RF014} - Ler casos de uso na ferramenta;
				\item \textbf{RF015} - Atualizar casos de uso na ferramenta;
				\item \textbf{RF016} - Remover casos de uso na ferramenta;
			\end{itemize}
		\item \textbf{Característica C1.1.2 - Manter metodologias ágeis}
			\begin{itemize}
				\item \textbf{RF017} - Cadastrar temas de investimento na ferramenta;
				\item \textbf{RF018} - Ler temas de investimento na ferramenta;
				\item \textbf{RF019} - Atualizar temas de investimento na ferramenta;
				\item \textbf{RF020} - Remover temas de investimento na ferramenta;
				\item \textbf{RF021} - Cadastrar épicos de negócio na ferramenta;
				\item \textbf{RF022} - Ler épicos de negócio na ferramenta;
				\item \textbf{RF023} - Atualizar épicos de negócio na ferramenta;
				\item \textbf{RF024} - Remover épicos de negócio na ferramenta;
				\item \textbf{RF025} - Cadastrar épicos arquiteturais na ferramenta;
				\item \textbf{RF026} - Ler épicos arquiteturais na ferramenta;
				\item \textbf{RF027} - Atualizar épicos arquiteturais na ferramenta;
				\item \textbf{RF028} - Remover épicos arquiteturais na ferramenta;
				\item \textbf{RF029} - Cadastrar features na ferramenta;
				\item \textbf{RF030} - Ler features na ferramenta;
				\item \textbf{RF031} - Atualizar features na ferramenta;
				\item \textbf{RF032} - Remover features na ferramenta;
				\item \textbf{RF033} - Cadastrar histórias de usuário na ferramenta;
				\item \textbf{RF034} - Ler histórias de usuário na ferramenta;
				\item \textbf{RF035} - Atualizar histórias de usuário na ferramenta;
				\item \textbf{RF036} - Remover histórias de usuário na ferramenta;
			\end{itemize}
		\item \textbf{Característica C1.2.1 - Manter informações sobre requisitos}
			\begin{itemize}
				\item \textbf{RF037} - Cadastrar atributos de usuário na ferramenta;
				\item \textbf{RF038} - Ler atributos de usuário na ferramenta;
				\item \textbf{RF039} - Atualizar atributos de usuário na ferramenta;
				\item \textbf{RF040} - Remover atributos de usuário na ferramenta;
				\item \textbf{RF041} - Relacionar atributos a requisitos;
				\item \textbf{RF042} - Alterar atributos dos requisitos;
				\item \textbf{RF043} - Gerar prioridade dos requisitos;
				\item \textbf{RF044} - Cadastrar cálculo para geração de prioridade;
			\end{itemize}
		\item \textbf{Característica C1.2.2 - Manter relação entre Requisitos}
			\begin{itemize}
				\item \textbf{RF045} - Cadastrar relação entre requisitos de níveis diferentes;
				\item \textbf{RF046} - Manter automáticamente relação entre requisitos de níveis diferentes quando ocorrer mudança;
				\item \textbf{RF047} - Manter relação de dependência entre requisitos do mesmo nível;
				\item \textbf{RF048} - Manter automáticamente relação de dependência entre requisitos do mesmo nível quando ocorrer mudança;
			\end{itemize}
		\item \textbf{Característica C1.3.1 - Auxiliar na escolha da metodologia}
			\begin{itemize}
				\item \textbf{RF049} - Fazer relatório com perguntas específicas para definir características do processo;
				\item \textbf{RF050} - Definir em qual metodologia o processo do cliente se encaixa;
			\end{itemize}
		\item \textbf{Característica C1.4.1 - Criar processos Híbridos}
			\begin{itemize}
				\item \textbf{RF051} - Integrar práticas de diferentes metodologias ao processo do usuário;
				\item \textbf{RF052} - Permitir alterações na metodologia sugerida;
				\item \textbf{RF053} - Impedir práticas equivalentes em metodologias diferentes pertencerem ao mesmo processo;
				\item \textbf{RF054} - Cadastrar requisitos no processo híbrido;
				\item \textbf{RF055} - Ler requisitos no processo híbrido;
				\item \textbf{RF056} - Atualizar requisitos no processo híbrido;
				\item \textbf{RF057} - Remover requisitos no processo híbrido;
			\end{itemize}
		\item \textbf{Característica C1.5.1 - Gerar e manter diagramas}
			\begin{itemize}
				\item \textbf{RF058} - Gerar \textit{diagrama de Ishikawa};
				\item \textbf{RF059} - Fazer alterações no \textit{diagrama de Ishikawa} gerado;
				\item \textbf{RF060} - Armazenar \textit{diagrama de Ishikawa};
				\item \textbf{RF061} - Gerar diagramas de casos de uso;
				\item \textbf{RF062} - Fazer alterações no diagrama de caso de uso gerado;
				\item \textbf{RF063} - Armazenar diagramas de casos deuso;
			\end{itemize}
		\subparagraph{Característica C1.5.2 - Controlar projeto por toda sua duração}
			\begin{itemize}
				\item \textbf{RF064} - Gerar plano de iteração em metodologias tradicionais;
				\item \textbf{RF065} - Gerar Backlog da sprint em metodologias ágeis;
				\item \textbf{RF066} - Cadastrar alterações no projeto;
				\item \textbf{RF067} - Permitir desfazer uma alteração a qualquer momento.
			\end{itemize}
	\end{enumerate}

%-----------------------------------------------------------------------------------------------------------
\subsection{Recursos do Produto}
\label{subsub:recursos_produto}

Os recursos do produto são as funcionalidades do sistema, o que futuramente será transformado em casos de uso, estes recursos estão organizados de acordo com a rastreabilidade proposta nas Figuras \ref{img:tabelaParte1} do relatório de projeto realizado anteriormente, e serão detalhados no documento de casos de uso, presente na Seção \ref{sec:documento_de_caso_de_uso} deste documento. Os recursos da Característica C1.1.1 estão organizados seguindo a rastreabilidade ilustrada na Figura \ref{img:tabelaParte1}.

\begin{figure}[H]
	\centering
	\includegraphics[width=0.8\textwidth]{imgModelagem/tabelaParte1}
	\caption{Tabela de rastreabilidade - 1 }
	\label{img:tabelaParte1}
\end{figure}

Seguem os recursos da característica C1.1.1, apresentados utilizando a ratreabilidade apresentada na Figura \ref{img:tabelaParte1}:

\subsubsection{Problema 1 - Falta de flexibilidade entre abordagens e ferramentas}

O problema 1, gera algumas necessidades, que serão colocadas a seguir.

\paragraph{Necessidade N1.1 - Utilização de ferramentas que se adequem as metodologias}

	\subparagraph{Característica C1.1.1 - Manter metodologias tradicionais}
		\begin{itemize}
			
			\item Caso de uso UC1.1.1.1 - Manter Problema;
				
				Requisitos funcionais associados: RF001, RF002, RF003, RF004.

				Este caso de uso será realizado pelo engenheiro de requisitos, e tem como objetivo \CRUD~ dos problemas na ferramenta.
			
			\item Caso de uso UC1.1.1.2 - Manter Necessidades;
				
				Requisitos funcionais associados: RF005, RF006, RF007, RF008.

				Este caso de uso será realizado pelo engenheiro de requisitos, e tem como objetivo o \CRUD~ das necessidades na ferramenta.
			
			\item Caso de uso UC1.1.1.3 - Manter Características;
				
				Requisitos funcionais associados: RF009, RF010, RF011, RF012.

				Este caso de uso será realizado pelo engenheiro de requisitos, e tem como objetivo o \CRUD~ das características na ferramenta.
			
			\item Caso de uso UC1.1.1.4 - Manter Casos de Uso.
				
				Requisitos funcionais associados: RF013, RF014, RF015, RF016.

				Este caso de uso será realizado pelo engenheiro de requisitos, e tem como objetivo o \CRUD~ dos casos de uso.
		\end{itemize}
	
	Os recursos da Característica C1.1.2 estão organizados seguindo a rastreabilidade ilustrada na Figura \ref{img:tabelaParte2}.

\begin{figure}[H]
	\centering
	\includegraphics[width=0.8\textwidth]{imgModelagem/tabelaParte2}
	\caption{Tabela de rastreabilidade - 2 }
	\label{img:tabelaParte2}
\end{figure}

	Seguem os recursos da característica C1.1.2, apresentados utilizando a ratreabilidade apresentada na Figura \ref{img:tabelaParte2}:

	\subparagraph{Característica C1.1.2 - Manter metodologias ágeis}
		\begin{itemize}
			
			\item Caso de uso UC1.1.2.1 - Manter Temas de Investimento;
				
				Requisitos funcionais associados: RF017, RF018, RF019, RF020.

				Este caso de uso será realizado pelo nível de portfólio do projeto, e tem como objetivo o \CRUD~ dos temas de investimento do projeto na ferramenta.
			
			\item Caso de uso UC1.1.2.2 - Manter Épicos;
				
				Requisitos funcionais associados: RF021, RF022, RF023, RF024, RF025, RF026, RF027, RF028.

				Este caso de uso será realizado pelo nível de portfólio do projeto, e tem como objetivo o \CRUD~ dos épicos do projeto na ferramenta.
			
			\item Caso de uso UC1.1.2.3 - Manter Features;
				
				Requisitos funcionais associados: RF029, RF030, RF031, RF032.

				Este caso de uso será realizado pelo nível de programa do projeto, e tem como objetivo o \CRUD~ das features do projeto na ferramenta.
			
			\item Caso de uso UC1.1.2.4 - Manter Histórias de Usuário.
				
				Requisitos funcionais associados: RF033, RF034, RF035, RF036.

				Este caso de uso será realizado pelo nível de time do projeto, e tem como objetivo o \CRUD~ dos temas de investimento do projeto na ferramenta.
		\end{itemize}

Os recursos da Característica C1.2.1 estão organizados seguindo a rastreabilidade ilustrada na Figura \ref{img:tabelaParte3}.

\begin{figure}[H]
	\centering
	\includegraphics[width=0.8\textwidth]{imgModelagem/tabelaParte3}
	\caption{Tabela de rastreabilidade - 3 }
	\label{img:tabelaParte3}
\end{figure}

	Seguem os recursos da Característica C1.2.1, apresentados utilizando a ratreabilidade apresentada na Figura \ref{img:tabelaParte3}:

\paragraph{Necessidade N1.2 - Apoio a utilização de uma rastreabilidade organizada e eficiente em qualquer abordagem}
	\subparagraph{Característica C1.2.1 - Manter informações sobre requisitos}
		\begin{itemize}
			
			\item Caso de uso UC1.2.1.1 - Manter Atributos;
			
			Requisitos funcionais associados: RF037, RF038, RF039, RF040, RF041, RF042.

			Este caso de uso será realizado ou pelo engenheiro de requisitos em uma metodologia tradicional, ou pelo nível de portfólio em metodologias ágeis, e tem como objetivo o \CRUD~ dos atributos no sistema, sendo atributos características dos requisitos.
		
			\item Caso de uso UC1.2.1.2 - Manter \textit{Roadmaps}.
			
			Requisitos funcionais associados: RF043, RF044.
			
			Este caso de uso será reaelizado ou pelo engenheiro de requisitos em uma metodologia tradicional, ou pela equipe de programa em metodologias ágeis, e tem como objetivo o \CRUD~ dos \textit{roadmaps} do projeto, sendo eles o uma priorização dos requisitos que devem ser implementados.

		\end{itemize}

		Os recursos da Característica C1.2.2 estão organizados seguindo a rastreabilidade ilustrada na Figura \ref{img:tabelaParte4}.

\begin{figure}[H]
	\centering
	\includegraphics[width=0.8\textwidth]{imgModelagem/tabelaParte4}
	\caption{Tabela de rastreabilidade - 4 }
	\label{img:tabelaParte4}
\end{figure}

	Seguem os recursos da característica C1.2.2, apresentados utilizando a ratreabilidade apresentada na Figura \ref{img:tabelaParte4}:

	\subparagraph{Característica C1.2.2 - Manter relação entre Requisitos}
		\begin{itemize}
			
			\item Caso de uso UC1.2.2.1 - Manter rastreabilidade horizontal entre os requisitos;
					
					Requisitos funcionais associados: RF045, RF046.

					Este caso de uso se refere ao ato do sistema manter as dependências dos requisitos com outros requisitos do mesmo nível, por exemplo um caso de uso que depende que um outro tenha sido feito já.
				
			
			\item Caso de uso UC1.2.2.2 - Manter rastreabilidade vertical entre os requisitos.
					
					Requisitos funcionais associados: RF047, RF048.

					Este caso de uso se refere ao ato do sistema manter as dependências dos requisitos com os requisitos de outros níveis, como por exemplo, quais necessidades estão ligadas a um problema.
		\end{itemize}

Os recursos das Necessidade N1.3, N1.4 e N1.5 estão organizados seguindo a rastreabilidade ilustrada na Figura \ref{img:tabelaParte5}.

\begin{figure}[H]
	\centering
	\includegraphics[width=0.8\textwidth]{imgModelagem/tabelaParte5}
	\caption{Tabela de rastreabilidade - 5 }
	\label{img:tabelaParte5}
\end{figure}

	Seguem os recursos da necessidade N1.3, N1.4 e N1.5, apresentados utilizando a ratreabilidade apresentada na Figura \ref{img:tabelaParte5}:

\paragraph{Necessidade N1.3 - Obter critérios fixos que direcionem o projeto para abordagem mais adequada}
	\subparagraph{Característica C1.3.1 - Auxiliar na escolha da metodologia}
		\begin{itemize}
			
			\item Caso de Uso UC1.3.1.1 - Definir Metodologia.
					
					Requisitos funcionais associados: RF049, RF050.

					Este caso de uso se refere ao ato do sistema definir a metodologia a ser adotada no projeto, de acordo com os dados entrados pelo engenheiro de requisitos ou equipe de portfólio.
		\end{itemize}

\paragraph{Necessidade N1.4 - Obter um processo de \er~ adaptável a qualquer abordagem}
	\subparagraph{Característica C1.4.1 - Criar processos Híbridos}
		\begin{itemize}
			
			\item Caso de uso UC1.4.1.1 - Definir ``hibridez'' do projeto;
					
					Requisitos funcionais associados: RF051, RF052, RF053.
					
					Este caso de uso se refere em montar uma metodologia própria para o projeto a ser desenvolvido, mantendo quais características de cada metodologia, ágil e tradicional, é mais adequada para cada característica do projeto.
			
			\item Caso de uso UC1.4.1.2 - Manter processos híbridos.
					
					Requisitos funcionais associados: RF054, RF055, RF056, RF057.
					
					Este caso de uso se refere ao ato do engenheiro de requisitos ou equipes das metodologias ágeis realizarem o \CRUD~ dos requisitos de projetos híbridos.
		\end{itemize}

\paragraph{Necessidade N1.5 - Gerar documentação de qualidade e fácil entendimento}
	\subparagraph{Característica C1.5.1 - Gerar e manter diagramas}
		\begin{itemize}
					
			\item Caso de uso UC1.5.1.1 - Gerar \textit{Diagrama de Ishikawa};
				
				Requisitos funcionais associados: RF058, RF059, RF060.

				Este caso de uso se refere a geração do \textit{Diagrama de Ishikawa} pelo engenheiro de requisitos para facilitar na identificação do problema a ser solucionado.
			
			\item Caso de uso UC1.5.1.2 - Gerar Diagramas de Casos de Uso.
				
				Requisitos funcionais associados: RF061, RF062, RF063.

				Este caso de uso se refere a geração do Diagrama de caso de uso por parte do engenheiro de requisitos em metodologias tradicionais.
		\end{itemize}

Os recursos das Característica C1.5.2 estão organizados seguindo a rastreabilidade ilustrada na Figura \ref{img:tabelaParte6}.

\begin{figure}[H]
	\centering
	\includegraphics[width=0.8\textwidth]{imgModelagem/tabelaParte6}
	\caption{Tabela de rastreabilidade - 6 }
	\label{img:tabelaParte6}
\end{figure}

	Seguem os recursos da Característica C1.5.2, apresentados utilizando a ratreabilidade apresentada na Figura \ref{img:tabelaParte5}:

	\subparagraph{Característica C1.5.2 - Controlar projeto por toda sua duração}
		\begin{itemize}
			
			\item Caso de uso UC1.5.2.1 - Gerar plano de iteração;
					
					Requisitos funcionais associados: RF064, RF065.
					
					Este caso de uso se refere a geração dos planos de iteração, tanto em metodologias ágeis, \textit{backlog}, quanto em metodologias tradicionais, o plano de iteração.

			\item Caso de uso UC1.5.2.2 - Controlar histórico de versão.
				
				Requisitos funcionais associados: RF066, RF067.

				Este caso de uso se refere ao controle das alterações realizadas nos projetos, e a possibilidade de voltar atrás em qualquer alteração, além do histórico de quem realizou cada alteração.

		\end{itemize}

%-----------------------------------------------------------------------------------------------------------
\subsection{Restrições}

\begin{itemize}
	\item \textbf{Técnica:}
		A ferramenta poderá ser executada pelos navegadores Google Chrome versão 37.0.2062.120 ou superior Firefox versão 33.0 ou superior, não sendo possível sua utilização no Internet Explorer ou Safari.
	\item \textbf{Tempo:}
		A primeira release da ferramenta deverá ser entregue até no máximo no dia 22 de novembro de 2014.
\end{itemize}

%-----------------------------------------------------------------------------------------------------------
\subsection{Requisitos não funcionais}

Requisitos não funcionais são características que não são funcionalidades do sistema em si, estão relacionados com aspectos como segurança, usabilidade, confiabilidade e performace \cite{derequisitos}.

A seguir estão listados os requisitos não funcionais do sistema em desenvolvimento.

\begin{itemize}
	\item \textbf{RNF01 - }A ferramenta necessitará de conexão com a internet;
	\item \textbf{RNF02 - }Deverá manter a segurança dos dados do sistema;
	\item \textbf{RNF03 - }Deverá ser possível acessá-la tanto de computadores como aparelhos móveis;
	\item \textbf{RNF04 - }A ferramenta deverá ser \opensource~ e sobre a licença \textit{GNU Affero General Public License} descrita em http://www.gnu.org/licenses/agpl-3.0.html;
	\item \textbf{RNF05 - }Durante o projeto apenas deverão ser usadas ferramentas \opensource.
\end{itemize}

%-----------------------------------------------------------------------------------------------------------

\subsection{Atributos do Recurso}

Atributos de recursos são basicamente descrições dos requisitos em alguma área em específico. Durante o projeto foram utilizados os atributos de arquitetura, prioridade e status, a tabela contendo estes atributos está ilustrada no relatório de projeto feito anteriormente.