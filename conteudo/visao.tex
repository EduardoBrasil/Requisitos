% !TEX root = ../main.tex

O documento de visão tem como objetivo definir uma visão geral do projeto, apresentar os problemas, os requisitos funcionais, não funcionanis, atores, entre outras informações que serão definidos com o cliente a fim de garantir que a equipe de desenvolvimento e o cliente estejam na maior sincronia possível \cite{IBM:2014:Online}.

%-----------------------------------------------------------------------------------------------------------
\subsection{Posicionando}
\subsubsection{Oportunidade de Negócios}

Atualmente, as ferramentas no mercado possuem limitações, como de qualidade, falta de flexibilidade na gerência, ou até mesmo o fechamento do código, que pode ser considerado uma limitação devida a redução de mão de obra para manutenção e evolução.

\subsubsection{Instrução do Problema}

A \er{} possui diversas \textit{rotas} possíveis para se seguir, como, por exemplo a rota ágil, tradicional ou até mesmo uma mistura das duas.

Infelizmente, cada ferramenta de gerência de requisitos é voltada para uma dessas possibilidades, tornando díficil a tarefa voltada para outras, gerando assim nos engenheiros de requisitos a necessidade de aprender a utilizar diversas ferramentas para poder organizar projetos com \textit{rotas} diferentes.

A utilização de apenas uma ferramenta que abrangesse as duas metodologias e ainda uma mistura das duas resolveria todo problema de gerência de requisitos em projetos que não se adequam a uma metodologia específica perfeitamente.

\subsubsection{Instrução de Posição do Produto}

Para os engenheiros de \er, a \nomeferramenta~ representará um avanço nas atividades de gerenciamento, pois os mesmos apenas precisarão aprender as funcionalidades de uma ferramenta, simplificando a mudança entre projetos que tomam rotas distintas.

%-----------------------------------------------------------------------------------------------------------
\subsection{Matriz de rastreabilidade de requisitos}

A matriz de rastreabilidade resume o modo como serão organizados os requisitos do sistema, e a escolhida para o projeto esta ilustrada na figura \ref{img:rastreabilidade}.

\begin{figure}[H]
	\centering
	\includegraphics[width=0.8\textwidth]{imgModelagem/diagrama}
	\caption{Matriz de rastreabilidade}
	\label{img:rastreabilidade}
\end{figure}

%-----------------------------------------------------------------------------------------------------------
\subsection{Descrições da Parte Interessada e do Usuário}

Nesta sessão serão identificados e detalhados os interessados e usuários da \nomeferramenta{}.

\subsubsection{Resumo da Parte Interessada e do Usuário}

Para melhor entendimento das características e responsabilidades dos interessados, utilizou-se uma tabela que apresenta todos os interessados no sistema, suas descrições, responsabilidades e os critérios de sucesso de suas funções na equipe, ilustrada na tabela \ref{tab:parteInteressada}. Com esta tabela, pode-se obter o entendimento necessário sobre os interessados e o quão importante eles são para o sucesso do sistema.

\begin{table}[htbp]
\centering
\begin{tabular}{|p{2cm}|p{5cm}|p{4cm}|p{4cm}|}
\hline
%-------------------------------------------------------
\textbf{Interessado} &
\textbf{Descrição} &
\textbf{Responsabilidade} &
\textbf{Critérios de Sucesso}
\\ \hline

%-------------------------------------------------------
Analista de Requisitos &
Membro da equipe de desenvolvimento com facilidade em comunicação, psicologia, sociologia, filosofia e mais áreas que possam facilitar a relação com o cliente. Seu conhecimento na área pode ser, dependendo da organização, baixo. &
Pessoa responsável por realizar a elicitação dos requisitos junto ao usuário. Deve elicitar os requisitos de forma adequada à garantir sucesso no desenvolvimento do \sw. &
Requisitos corretamente elicitados e prontos para serem documentados. 
\\ \hline
%-------------------------------------------------------
Gerente de Requisitos &
Conhecedor de todo o processo de desenvolvimento e com contato frequente com o cliente. Seu conhecimento deve ser alto. &
Pessoa responsável por administrar os requisitos durante todo processo de desenvolvimento de \sw, garantindo o mínimo esforço em casos de mudança de requisitos. &
Requisitos bem administrados para, no caso de mudanças nos requisitos, existir o menor impacto possível na equipe de desenvolvimento.
\\ \hline
%--------------------------------------------------------
Programador &
Pessoa com capacidade em linguagens e lógica de programação &
Implementar o sistema utilizando as técnologias definidas &
Implementação do sistema de acordo com os requisitos levantados e cadastrados na ferramenta
\\ \hline

\end{tabular}
\label{}
\caption{Parte Interessada}
\label{tab:parteInteressada}
\end{table}

\subsubsection{Principais Problemas e Necessidades da Parte Interessada}

O problema a ser resolvido pela \nomeferramenta~ deve estar bastante claro entre todos os \stakeholder, para que o desenvolvimento passe pela menor quantidade possível de dificuldades quanto ao entendimento de onde focar esforços para desenvolver a solução.

Para o mapeamento do problema principal e suas causas, foi utilizada a técnica do \textit{Diagrama de Ishikawa}, que se encontra na figura \ref{img:fishbone}.

%-------------------------------------FISHBONE AQUI----------------------------------------
\begin{figure}[H]
	\centering
	\includegraphics[width=0.8\textwidth]{imgModelagem/fishbone}
	\caption{Diagrama de Ishikawa}
	\label{img:fishbone}
\end{figure}
%-------------------------------------FISHBONE AQUI----------------------------------------


%-------------------------------------FRAMEWORK DE Problema-----------------------------------
Para melhor entendimento do problema, utilizamos a técnica de \textit{Framework de problema}, que consiste em criar uma tabela apresentando o problema, os afetados, o impacto e qual seria uma solução bem sucedida, o Framework está retratado na tabela \ref{tab:frameworkproblema}.

\begin{table}[htbp]
\centering
\begin{tabular}{|p{3cm}|p{10cm}|p{2.5cm}|}
%----------------------------------------------
\hline
\textbf{O Problema:} &
Falta de flexibilidade entre Abordagens e Ferramentas. 
\\ \hline
%----------------------------------------------
\textbf{Afeta:} &
Todos os desenvolvedores de \sw~que necessitam de uma flexibilidade maior na gerência de requisitos.
\\ \hline
%----------------------------------------------
\textbf{Cujo impacto é:} &
Processo de requisitos mal gerenciados, aumentando a possibilidade de erros durante o desenvolvimento.
\\ \hline
%----------------------------------------------
\textbf{Uma solução bem sucedida seria:} &
Utilização de uma ferramenta que faça gerência de requisitos de forma flexivel, podendo utilizá-la em qualquer metodologia.
\\ \hline
%----------------------------------------------
\end{tabular}
\caption{Framework de Problema}
\label{tab:frameworkproblema}
\end{table}

%-----------------------------------------FRAMEWORK DE NECESSIDADES----------------------------------------
Após o entendimento do problema, vê-se necessária a documentação das necessidades do cliente. Utilizou-se de uma técnica chamada \textit{framework de necessidades} na qual são apresentados todos os problemas, as necessidades, a solução atual e a solução proposta. Dessa forma, pode-se obter um entendimento mais organizado dos problemas e necessidades do cliente, de acordo com o retradado na tabela \ref{tab:frameworknecessidade}.

\begin{table}[H]
\centering
\begin{tabular}{|p{5cm}|p{3cm}|p{3cm}|p{5cm}|}

%-------------------------------------------------------------
\hline
\textbf{Necessidade} &
\textbf{Problema} &
\textbf{Solução Atual} &
\textbf{Solução Proposta}
\\ \hline

%-------------------------------------------------------------
Utilização de ferramentas que se adequem as metodologias. &
Ferramentas associadas a metodologias específicas. &
Equipe utiliza mais de uma ferramenta para abranger as abordagens utilizadas. &
Criação de uma ferramenta que seja flexível para qualquer metodologia, abrangendo todas as agordagens e até a mesclagem das mesmas.
\\ \hline
%-------------------------------------------------------------
Apoio a utilização de uma rastreabilidade organizada e eficiente em qualquer abordagem. &
Limitação de ratreabilidade. &
A equipe precisa criar sua rastreabilidade sem o apoio de uma ferramenta flexível. &
Criação de uma ferramenta que gere a rastreabilidade dos requisitos de forma organizada e eficiente para qualquer abordagem.
\\ \hline
%-------------------------------------------------------------
Obter critérios fixos que redirecionem o projeto para a abordagem mais adequada. &
Falta de critérios para seleção de abordagens. &
Equipe precisa estudar as características do projeto e decidir qual a abordagem mais adequada. &
Criação de uma ferramenta que recolha as características do projeto e apresente a abordagem mais adequada.
\\ \hline
%-------------------------------------------------------------
Obter um processo de \textit{E.R.} adaptável a qualquer abordagem. &
Falta de definição de um processo de \textit{E.R.} adaptável. &
Utilização de um processo inflexível e voltado apenas para uma abordagem. &
Criação de uma ferramenta que gerencie processos flexíveis.
\\ \hline
%-------------------------------------------------------------
Gerar documentação de qualidade e fácil entendimento &
Dificuldade em gerar documentação para pessoas de fora da equipe &
Utilização de ferramentas a parte para gerar Diagramas de Caso de Uso e \textit{Diagramas de Ishikawa}. &
Integração da documentação na ferramenta de requisitos.
\\ \hline
%-------------------------------------------------------------
\end{tabular}
\caption{Framework de Necessidades}
\label{tab:frameworknecessidade}
\end{table}

%-----------------------------------------------------------------------------------------------------------
\subsection{Visão Geral do Produto}
	
Nesta seção, pode-se ter um entendimento geral de como será o produto final, quais serão suas características, como serão suas funcionalidades e etc.

\subsubsection{Perspectiva do Produto}
	
O produto se encontrará em um contexto onde existem inúmeras ferramentas com o mesmo propósito, porém, as ferramentas existentes são inflexíveis quando se trata da abordagem que será seguida durante o desenvolvimento de \sw. Esta falha será corrigida na \nomeferramenta, que irá propor uma metodologia para cada projeto em particular de acordo com suas características.

A ferramenta pode ser autocontida, não necessitando do apoio de nenhum outro sistema, porém a utilização de ferramentas de modelagem de processos é bastante indicada para que a máxima organização do projeto seja alcançada.

\subsubsection{Resumo das Capacidades}
	
O grande diferencial da \nomeferramenta~ será a flexibilização na abordagem que será seguida durante o gerenciamento de projetos de \sw. O sistema deverá indicar a melhor abordagem a ser seguida pela equipe de desenvolvimento, garantindo a otimização do processo de desenvolvimento.

A ferramenta será capaz de disponibilizar a opção de modificar a abordagem indicada pela ferramenta, para que a equipe de desenvolvimento possa escolher a abordagem na qual os mesmos se sentem mais a vontade.

%-----------------------------------------------------------------------------------------------------------
\subsection{Recursos do Produto}
\label{subsub:recursos_produto}

Os recursos do produto são as funcionalidades do sistema, o que futuramente será transformado em casos de uso, estes recursos estão organizados de acordo com a rastreabilidade proposta na figura \ref{img:rastreabilidade}, e serão detalhados no documento de casos de uso, presente na sessão \ref{sec:documento_de_caso_de_uso} deste documento. Os seguintes atributos serão organizados da forma que se apresenta na \ref{img:tabelaRastreabilidade}.

\begin{figure}[H]
	\centering
	\includegraphics[width=0.8\textwidth]{imgModelagem/tabelaRastreabilidade}
	\caption{Tabela de rastreabilidade}
	\label{img:tabelaRastreabilidade}
\end{figure}

Seguem os recursos do sistema, apresentados utilizando a ratreabilidade apresentava na figura \ref{img:tabelaRastreabilidade}:

\subsubsection{Problema 1 - Falta de flexibilidade entre abordagens e ferramentas}

O problema 1, gera algumas necessidades, que serão colocadas a seguir.

\paragraph{Necessidade N1.1 - Utilização de ferramentas que se adequem as metodologias}

	\subparagraph{Característica C1.1.1 - Manter metodologias tradicionais}
		\begin{itemize}
			\item Caso de uso UC1.1.1.1 - Manter Problema;\\
				Este caso de uso será realizado pelo engenheiro de requisitos, e tem como objetivo \CRUD~ dos problemas na ferramenta.
			\item Caso de uso UC1.1.1.2 - Manter Necessidades;\\
				Este caso de uso será realizado pelo engenheiro de requisitos, e tem como objetivo o \CRUD~ das necessidades na ferramenta.
			\item Caso de uso UC1.1.1.3 - Manter Características;\\
				Este caso de uso será realizado pelo engenheiro de requisitos, e tem como objetivo o \CRUD~ das características na ferramenta.
			\item Caso de uso UC1.1.1.4 - Manter Casos de Uso.\\
				Este caso de uso será realizado pelo engenheiro de requisitos, e tem como objetivo o \CRUD~ dos casos de uso.
		\end{itemize}
	
	\subparagraph{Característica C1.1.2 - Manter metodologias ágeis}
		\begin{itemize}
			\item Caso de uso UC1.1.2.1 - Manter Temas de Investimento;\\
				Este caso de uso será realizado pelo nível de portfólio do projeto, e tem como objetivo o \CRUD~ dos temas de investimento do projeto na ferramenta.
			\item Caso de uso UC1.1.2.2 - Manter Épicos;\\
				Este caso de uso será realizado pelo nível de portfólio do projeto, e tem como objetivo o \CRUD~ dos épicos do projeto na ferramenta.
			\item Caso de uso UC1.1.2.3 - Manter Features;\\
				Este caso de uso será realizado pelo nível de programa do projeto, e tem como objetivo o \CRUD~ das features do projeto na ferramenta.
			\item Caso de uso UC1.1.2.4 - Manter Histórias de Usuário.\\
				Este caso de uso será realizado pelo nível de time do projeto, e tem como objetivo o \CRUD~ dos temas de investimento do projeto na ferramenta.
		\end{itemize}

\paragraph{Necessidade N1.2 - Apoio a utilização de uma rastreabilidade organizada e eficiente em qualquer abordagem}
	\subparagraph{Característica C1.2.1 - Manter informações sobre requisitos}
		\begin{itemize}
			\item Caso de uso UC1.2.1.1 - Manter Atributos;\\
				Este caso de uso será realizado ou pelo engenheiro de requisitos em uma metodologia tradicional, ou pelo nível de portfólio em metodologias ágeis, e tem como objetivo o \CRUD~ dos atributos no sistema, sendo atributos características dos requisitos.
			\item Caso de uso UC1.2.1.2 - Manter \textit{Roadmaps}.\\
				Este caso de uso será reaelizado ou pelo engenheiro de requisitos em uma metodologia tradicional, ou pelo nível de portfólio em metodologias ágeis, e tem como objetivo o \CRUD~ dos \textit{roadmaps} do projeto, sendo eles o uma priorização dos requisitos que devem ser implementados.

		\end{itemize}

	\subparagraph{Característica C1.2.2 - Manter relação entre Requisitos}
		\begin{itemize}
			\item Caso de uso UC1.2.2.1 - Manter rastreabilidade horizontal entre os requisitos;\\
				Este caso de uso se refere ao ato do sistema manter as dependências dos requisitos com outros requisitos do mesmo nível, por exemplo um caso de uso que depende que um outro tenha sido feito já.
			\item Caso de uso UC1.2.2.2 - Manter rastreabilidade vertical entre os requisitos.\\
				Este caso de uso se refere ao ato do sistema manter as dependências dos requisitos com os requisitos de outros níveis, como por exemplo, quais necessidades estão ligadas a um problema.
		\end{itemize}

\paragraph{Necessidade N1.3 - Obter critérios fixos que direcionem o projeto para abordagem mais adequada}
	\subparagraph{Característica C1.3.1 - Auxiliar na escolha da metodologia}
		\begin{itemize}
			\item Caso de Uso UC1.3.1.1 - Definir Metodologia.\\
				Este caso de uso se refere ao ato do sistema definir a metodologia a ser adotada no projeto, de acordo com os dados entrados pelo engenheiro de requisitos.
		\end{itemize}

\paragraph{Necessidade N1.4 - Obter um processo de \er~ adaptável a qualquer abordagem}
	\subparagraph{Característica C1.4.1 - Criar processos Híbridos}
		\begin{itemize}
			\item Caso de uso UC1.4.1.1 - Definir ``hibridez'' do projeto;\\
				Este caso de uso se refere em montar uma metodologia própria para o projeto a ser desenvolvido, mantendo quais características de cada metodologia, ágil e tradicional, é mais adequada para cada característica do projeto.
			\item Caso de uso UC1.4.1.2 - Manter processos híbridos.\\
				Este caso de uso se refere ao ato do engenheiro de requisitos realizar o \CRUD~ dos requisitos de projetos híbridos.
		\end{itemize}

\paragraph{Necessidade N1.5 - Gerar documentação de qualidade e fácil entendimento}
	\subparagraph{Característica C1.5.1 - Gerar e manter diagramas}
		\begin{itemize}
			\item Caso de uso UC1.5.1.1 - Gerar \textit{Diagrama de Ishikawa};\\
				Este caso de uso se refere a geração do \textit{Diagrama de Ishikawa} pelo engenheiro de requisitos para facilitar na identificação do problema a ser solucionado. 
			\item Caso de uso UC1.5.1.2 - Gerar Diagramas de Casos de Uso.\\
				Este caso de uso se refere a geração do Diagrama de caso de uso por parte do engenheiro de requisitos em metodologias tradicionais.
		\end{itemize}

	\subparagraph{Característica C1.5.2 - Controlar projeto por toda sua duração}
		\begin{enumerate}
			\item Caso de uso UC1.5.2.1 - Gerar plano de iteração;\\
				Este caso de uso se refere a geração dos planos de iteração, tanto em metodologias ágeis, \textit{backlog}, quanto em metodologias tradicionais, o plano de iteração.
			\item Caso de uso UC1.5.2.2 - Controlar histórico de versão.\\
				Este caso de uso se refere ao controle das alterações realizadas nos projetos, e a possibilidade de voltar atrás em qualquer alteração, além do histórico de quem realizou cada alteração.
		\end{enumerate}

%-----------------------------------------------------------------------------------------------------------
\subsection{Restrições}

\begin{itemize}
	\item Técnica
		A ferramenta poderá ser executada pelos navegadores Google Chrome versão 37.0.2062.120 ou superior Firefox versão 33.0 ou superior, não sendo possível sua utilização no Internet Explorer ou Safari.
	\item Tempo
		A primeira release da ferramenta deverá ser entregue até no máximo no dia 22 de novembro de 2014.
\end{itemize}

%-----------------------------------------------------------------------------------------------------------
\subsection{Requisitos não funcionais}

Requisitos não funcionais são características que não são funcionalidades do sistema em si, estão relacionados com aspectos como segurança, usabilidade, confiabilidade e performace \cite{derequisitos}.

A seguir estão listados os requisitos não funcionais do sistema em desenvolvimento.

\begin{itemize}
	\item A ferramenta necessitará de conexão com a internet;
	\item Deverá manter a segurança dos dados do sistema;
	\item Deverá ser possível acessá-la tanto de computadores como aparelhos móveis;
	\item A ferramenta deverá ser \opensource~ e sobre a licença \textit{GNU Affero General Public License} descrita em http://www.gnu.org/licenses/agpl-3.0.html;
	\item Durante o projeto apenas deverão ser usados ferramentas \opensource.
\end{itemize}

%-----------------------------------------------------------------------------------------------------------

\subsection{Atributos do Recurso}

Atributos de recursos são basicamente descrições dos requisitos em alguma área em específico. Durante o projeto foram utilizados os atributos de arquitetura, prioridade e status.

\subsubsection{Atributos de Arquitetura}

Atributos de arquitetura são atributos que definem a complexidade arquitetural de se implementar algum requisito, por exemplo, se haverá necessidade de alterar arquitetura do software, e estão retratados na tabela \ref{tab:atributo_arquitetura}

\begin{table}[H]
\begin{tabular}{|p{4cm}|p{11cm}|}
%-------------------------------------------
\hline
\textbf{Atributo} &
\textbf{Descrição}
\\ \hline

%-------------------------------------------
\textbf{Grande} &
Para implementar o requisito, a arquitetura sofrerá uma grande alteração
\\ \hline

%-------------------------------------------
\textbf{Média} &
A arquitetura terá uma alteração consideravel na implementação do requisito
\\ \hline

%-------------------------------------------
\textbf{Baixa} &
A arquitetura terá uma pequena alteração para sustentar o requisito
\\ \hline

%-------------------------------------------
\textbf{Nenhuma} &
O requisito não terá impacto nenhum na arquitetura do projeto
\\ \hline
\end{tabular}
\caption{Atributo de Arquitetura}
\label{tab:atributo_arquitetura}
\end{table}

\subsubsection{Atributo de Prioridade}

Atributos de prioridade são atributos que definem o quão importante um requisito é para o cliente, definindo se ele deve ser implementado o mais rápido possível ou se pode ter sua implementação adiada, estão retratados na tabela \ref{tab:atributo_prioridade}.

\begin{table}[H]
\begin{tabular}{|p{4cm}|p{11cm}|}

\hline
\textbf{Atributo} &
\textbf{Descrição}
\\ \hline

%-----------------------------------------
\textbf{Alta prioridade} &
Os requisitos marcados com este atributo são requisitos que possuem um grande interesse do cliente
\\ \hline

%-----------------------------------------
\textbf{Média prioridade} &
Os requisitos marcados por este atributo são requisitos que o cliente possui um grande interesse, porém não existe necessidade de implementá-lo rapidamente.
\\ \hline

%----------------------------------------
\textbf{Baixa prioridade} &
Os requisitos marcados por este atributo são requisitos que o cliente deseja, porém não são essenciais para o funcionamento da solução.
\\ \hline

\end{tabular}
\caption{Atributo de prioridade}
\label{tab:atributo_prioridade}
\end{table}

\subsubsection{Atributos de Status}

Atributos de status são atributos que indicam em que fase um requisito está, de acordo com a tabela a seguir, e estão retrados na tabela \ref{tab:atributo_status}.

\begin{table}[H]
\begin{tabular}{|p{4cm}|p{11cm}|}

\hline
\textbf{Atributo} &
\textbf{Descrição}
\\ \hline

%-------------------------------
\textbf{Aceito} &
Requisito devidamente implementado e aceito pelo cliente
\\ \hline

%-------------------------------
\textbf{Implementado} &
Requisito implementado porém esperando aceitação
\\ \hline

%-------------------------------
\textbf{Detalhado} &
Requisito detalhado, esperando por implementação
\\ \hline

%------------------------------
\textbf{Elicitação aceita} &
Requisito elicitado e aceito pelo cliente porém sem detalhamento
\\ \hline

%------------------------------
\textbf{Elicitado} &
Elicitado porém esperando aceitação do cliente
\\ \hline

\end{tabular}
\caption{Atributo de status}
\label{tab:atributo_status}
\end{table}