% !TEX root = ../main.tex

Cada uma das rotas possíveis para o desenvolvimento de \sw, Ágil e Tradicional, possuem características bem definidas, por isso em alguns casos é necessário a utilização de algumas práticas de cada uma delas para poder otimizar o processo.

Afim de chegar à melhor prática possível para o projeto a seguir foi desenvolvida uma tabela com cada uma das características de projeto, equipe e negócio, analisando cada uma delas para identificar em qual das duas rotas o projeto se encaixa melhor, e o resultado foi a tabela \ref{tab:caracteristcas_do_projeto}.

\begin{table}[h]
    \begin{tabular}{|p{2cm}|p{5cm}|c|c|p{4cm}|}
        %------- cabeçalho--------
        \hline
        \textbf{Itens} &
        \textbf{Características} &
        \textbf{Tradicional} &
        \textbf{Ágil} &
        \textbf{Descrição}
        \\ 
        %-------------------------
        \hline
        \multirow{2}{*}{
            %-------------------------
            \textbf{Projeto}} &
                %-------------------------
                Entregas parciais &
                 &
                x &
                Utilização de Sprint para entregar \sw~ funcional em partes.
                \\
                \cline{2-5} 
                %-------------------------
                 &
                Mudança de equipe de desenvolvimento &
                x &
                 &
                Documentação do projeto necessária  ara apresentar para a nova equipe de desenvolvimento. 
                \\ 
                \hline
            %-------------------------
        %-------------------------
        \multirow{3}{*}{
            %-------------------------
            \textbf{Equipe}} &
                %-------------------------
                Reuniões frequentes com o cliente &
                 &
                x &
                Cliente faz parte da equipe de desenvolvimento, com reuniões semanais
                \\
                \cline{2-5}
                %------------------------- 
                 &
                Maior afinidade com metodologia ágil &
                 &
                x &
                A equipe prefere desenvolver em ágil
                \\
                \cline{2-5} 
                %-------------------------
                 &
                Equipe pequena &
                 &
                x &
                A equipe conhece todo o código programação em pares
                \\
                \hline
            %-------------------------
        %-------------------------
        \multirow{2}{*}{
            %-------------------------
            \textbf{Negócio}}
                %-------------------------
                 &
                Requisitos mutáveis &
                 &
                x &
                Provável evolução do sistema após o fim da primeira etapa de projeto
                \\
                \cline{2-5}
                %-------------------------
                 &
                Documentação extensiva para manter o sistema &
                x &
                 &
                Utlização de metodologia tradicional para documentação do projeto
                \\
                \hline
            %-------------------------
        %-------------------------
    \end{tabular}
    \caption{Tabela de características do projeto}
    \label{tab:caracteristcas_do_projeto}
\end{table}

O trabalho será feito em cima de uma metodologia Híbrida, onde a rastreabilidade e a definição dos requisitos será feita com base no método 
tradicional, contendo:

\begin{itemize}
    \item{Problemas}
    \item{Necessidades}
    \item{Características}
    \item{Casos de uso}
\end{itemize}

A escolha da metodologia Tradicional nesta etapa, se originou pelo fato de que o projeto em si está sendo tratado como um projeto grande, onde uma documentação e uma análise inicial do contexto de negocio mais completa trariam maiores benefícios, como uma previsibilidade de recursos, estabilidade do processo, assim como uma alta garantia das funcionalidades do projeto.

No quesito de extensa documentação, para o desenvolvimento deste projeto é um ponto forte devido ao fato de que provavelmente a equipe atual, com o tempo, será substituída por uma nova.

Porém, como a equipe de desenvolvimento é pequena, possui um conhecimento maior em desenvolvimento ágil e o PO possui bastante tempo disponível para reuniões e desenvolvimento junto da equipe, o desenvolvimento do projeto e o nível de aproximação cliente/equipe seguirão a metodologia ágil, buscando maior garantia de qualidade de desenvolvimento.

Outro ponto importante a ser frizado é a possibilidade de que os requisitos mudem após a primeira etapa de desenvolvimento, já que o sistema que será implementado durante a disciplina de \hell~ abrangerá apenas uma pequena parte do sistema completo. 
    
Para o desenvolvimento do sistema, alguns valores da metodologia ágil serão utilizados. Tais como:

\begin{itemize}
    \item{Responder a mudanças}
    \item{Colaboração com cliente}
    \item{Entrega contínua de \sw}
    \item{Em intervalor regulares, refletir em como ficar mais efetivo}
    \item{Desenvolvimento dividido em Sprints de 15 dias cada}
    \item{Desenvolvimento em pares para nivelar o conhecimento da equipe}
\end{itemize}

A definição destas atividades foi devida à possibilidade de frequentes reuniões entre a equipe e o cliente. 

Para este projeto estamos seguindo padrão definido por \cite{cho2009hybrid} colocado a seguir:

\begin{itemize}
    \item{Através dos indivíduos e interações, definir um processo para o decorrer do projeto}
    \item{Através de documentação, ter \sw~ funcional que agregue valor ao cliente}
    \item{Colaboração do cliente antes de negociações e contratos}
    \item{Através de um plano, responder as mudanças até a chegada ao destino}
\end{itemize}
