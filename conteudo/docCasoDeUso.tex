% !TEX root = ../main.tex

O documento de casos de uso tem como finalidade o detalhamento a fundo dos recursos do programa, aqui chamados de caso de uso, listados na sessão \ref{subsub:recursos_produto} deste documento, colocando todas as suas características, restrições e caminhos possíveis.

\subsection{Identificação dos atores}

Neste contexto, atores são representações genéricas de usuários do sistema, podende ser qualquer utilizador, sem se preocupar com nome do executor, apenas com sua função, esses atores estão listados na tabela \ref{tab:atores}.

\begin{table}[H]
\centering
\begin{tabular}{|l|p{8cm}|}

\hline
\textbf{Atores} &
\textbf{Descrição}
\\ \hline
%----------------------------------------------
Engenheiro de Requisitos &
Responsável, em metodologias tradicionais, pela elicitação dos requisitos e pela manutenção da rastreabilidade do sistema
\\ \hline
%----------------------------------------------
Analista de Requisitos &
Responsável, em metodologias tradicionais, gerência dos requisitos
\\ \hline
%----------------------------------------------
\textit{Product Owner} &
Responsável, em metodologias ágeis, por escrever histórias de usuário
\\ \hline
%----------------------------------------------
Equipe de portfólio &
Responsável, em metodologias ágeis, por gerir a parte do portfólio, como temas de investimento e épicos
\\ \hline
%----------------------------------------------
Equipe de programa &
Responsável, em metodologias ágeis, por gerir as features do sistema e manter a entrega das releases em dia
\\ \hline
%----------------------------------------------
Time &
Responsável, em metodologias ágeis, por implementar as histórias de usuário
\\ \hline
%----------------------------------------------
Cliente &
Responsável por validar os requisitos
\\ \hline

\end{tabular}
\caption{Atores do sistema}
\label{tab:atores}
\end{table}

\subsection{Diagrama de casos de uso}

\subsection{Detalhamento dos casos de uso}

Detalhamentos de casos de uso seve para definir o que cada caso de uso fará, quem irá realizá-lo, e como ele irá responder a falhas caso haja, e todos os seus caminhos possíveis.

A seguir estão os detalhamentos dos casos de uso que serão implementados nas sprints 1 e 2, detalhadas na tabela \ref{tab:primeiro_roadmap}.

\subsubsection{Caso de Uso - UC1.3.1.1 - Definir Metodologia}

\paragraph{Descrição}

Este caso de uso especifica a ação do sistema de, dada as informações solicitadas, selecionar a melhor rota possível para o desenvolvimento do projeto, podendo o usuário, ao final do questionário, decidir se irá seguir ou não a rota sugerida, e então preparar a ferramenta para a metodologia escolhida.

\begin{enumerate}
	\item Atores
		Engenheiro de requisitos. 
	\item Pré-condições
		Não existe pré condições para este caso de uso.
	\item Pós-condições
		A rota a ser utilizada deve estar definida ao final da execução deste caso de uso.
\end{enumerate}

\paragraph{Fluxo básico}

	\begin{enumerate}
		\item Ator decide criar um novo projeto;
		\item Sistema apresenta um questionário para recolher informações doprojeto;
			As perguntas são:
			\begin{itemize}
				\item
				\item
				\item
			\end{itemize}
		\item Ator responde questionário;
		\item Sistema calcula estisticamente qual rota deve ser utilizada, de acorodo com as respostas do ator;
			\label{item:1.3.1.1_empate}
		\item Sistema apresenta ao ator a escolha da metodologia;
		\item Usuário aceita a metodologia;
			\label{item:1.3.1.1_recusado}
		\item Sistema prepara a ferramenta para utilização da metodologia escolhida.
			\label{item:1.3.1.1_retorno1}
	\end{enumerate}

\paragraph{Fluxo alternativo A}

	\begin{enumerate}
		\item No passo \ref{item:1.3.1.1_empate} do fluxo básico, caso haja um empate entre metodologias;
		\item Sistema apresenta aos ao ator as metodologias empatadas e suas características;
		\item Ator escolhe a metodologia que deseja;
		\item O fluxo retorna para o passo \ref{item:1.3.1.1_retorno1} do fluxo básico.
	\end{enumerate}

\paragraph{Fluxo alternativo B}

	\begin{itemize}
		\item No passo \ref{item:1.3.1.1_recusado} do fluxo básico, caso o ator não aceite a metodologia proposta pelo sistema;
		\item Ator rejeita a opção da metodologia escolhida pelo sistema;
		\item Sistema apresenta todas as opções de metodologias cadastradas para que o usuário possa escolher;
		\item retorna para o passo \ref{item:1.3.1.1_retorno1} do fluxo básico.
	\end{itemize}	