% !TEX root = ../main.tex

O documento de casos de uso tem como finalidade o detalhamento a fundo dos recursos do programa, aqui chamados de caso de uso, listados na Sessão \ref{subsub:recursos_produto} deste documento, colocando todas as suas características, restrições e caminhos possíveis.

\subsection{Identificação dos atores}

Neste contexto, atores são representações genéricas de usuários do sistema, podendo ser qualquer utilizador, sem se preocupar com nome do executor, apenas com sua função, esses atores estão listados na Tabela \ref{tab:atores}.

\begin{table}[H]
\centering
\begin{tabular}{|l|p{8cm}|}

\hline
\textbf{Atores} &
\textbf{Descrição}
\\ \hline
%----------------------------------------------
Engenheiro de Requisitos &
Responsável, em metodologias tradicionais, pela elicitação dos requisitos e pela manutenção da rastreabilidade do sistema
\\ \hline
%----------------------------------------------
Analista de Requisitos &
Responsável, em metodologias tradicionais, gerência dos requisitos
\\ \hline
%----------------------------------------------
\textit{Product Owner} &
Responsável, em metodologias ágeis, por escrever histórias de usuário
\\ \hline
%----------------------------------------------
Equipe de portfólio &
Responsável, em metodologias ágeis, por gerir a parte do portfólio, como temas de investimento e épicos
\\ \hline
%----------------------------------------------
Equipe de programa &
Responsável, em metodologias ágeis, por gerir as features do sistema e manter a entrega das releases em dia
\\ \hline
%----------------------------------------------
Time &
Responsável, em metodologias ágeis, por implementar as histórias de usuário
\\ \hline
%----------------------------------------------
Cliente &
Responsável por validar os requisitos
\\ \hline

\end{tabular}
\caption{Atores do sistema}
\label{tab:atores}
\end{table}

\subsection{Diagrama de casos de uso}

Uma forma de apresentar os recursos do sistema de forma clara e objetiva é com a utilização de Diagramas de Casos de Uso, os quais apresentam todos os Casos de Uso do sistema e qual sua interação com os atores do sistema. Com a utilização deste diagrama, pode-se obter o entendimento sobre o que cada ator poderá fazer ao utilizar a ferramenta.

O diagrama está representado na figura X:

\subsection{Detalhamento dos casos de uso}

Detalhamentos de casos de uso seve para definir o que cada caso de uso fará, quem irá realizá-lo, e como ele irá responder a falhas caso haja, e todos os seus caminhos possíveis.

A seguir estão os detalhamentos dos casos de uso que serão implementados nas sprints 1 e 2, detalhadas na Tabela \ref{tab:primeiro_roadmap}.

\subsubsection{Caso de Uso - UC1.3.1.1 - Definir Metodologia}

\paragraph{Descrição}

Este caso de uso especifica a ação do sistema de, dada as informações solicitadas, selecionar a melhor rota possível para o desenvolvimento do projeto, podendo o usuário, ao final do questionário, decidir se irá seguir ou não a rota sugerida, e então preparar a ferramenta para a metodologia escolhida.

\begin{enumerate}
	\item Atores
		Engenheiro de requisitos. 
	\item Pré-condições
		Não existe pré condições para este caso de uso.
	\item Pós-condições
		A rota a ser utilizada deve estar definida ao final da execução deste caso de uso.
	\item Requisitos Funcionais
		\begin{itemize}
			\item RF02 - O sistema deverá calcular a metodologia idela para o projeto;
			\item RF03 - O sistema deverá permitir ao usuário escolher a própria metodologia;
			\item RF07 - O sistema deverá ter um menu dropdown em todas as páginas para o usuário acessar seus projetos;
			\item RF08 - O sistema deverá ter um link em todas as páginas para criação de novos projetos;
			\item RF10 - O sistema deverá gerar metodologias novas caso necessário.
		\end{itemize}
	\item Requisitos Não Funcionais
		\begin{itemize}
			\item RNF01 - O usuário não deve necessitar de treinamento para utilizar a ferramenta;
			\item RNF02 - O usuário não deve levar mais de um mês para estar totalmente produtivo na utilização da ferramenta.
			\item RNF07 - O sistema deve ter um tempo de resposta em qualquer página de no máximo dois segundos e em média um segundo;
			\item RNF08 - O sistema deve ter no máximo 1 consulta ao banco de dados por funcionalidade;
		\end{itemize}
\end{enumerate}

\paragraph{Fluxo básico}

	\begin{enumerate}
		\item Ator decide criar um novo projeto;
		\item Sistema apresenta um questionário para recolher informações doprojeto;
			As perguntas são:
			\begin{enumerate}
				\item \textbf{Equipe};
					\begin{enumerate}
						\item Qual o tamanho de sua equipe;

							Equipes grandes geralmente possuem uma dificuldade maior em se implementar metodologias ágeis.

						\item Qual a probabilidade da equipe vir a mudar durante o desenvolvimento;

							Em casos no qual a equipe vai mudar durante o projeto, as metodologias ágeis falham em manter a documentação necessária para novos membros se atualizarem de como anda o projeto.

						\item Qual será a disponibilidade do seu cliente durante o projeto;

							Em casos no qual o cliente não pode estar presente constantemente no projeto, abordagens ágeis podem vir a ser falhas por haver dificuldades em elicitar requisitos novos e de priorizar os já elicitados.

						\item Quão entrosada é sua equipe;

							Metodologia ágeis requerem uma equipe auto gerenciável e esteja bem entrosada evitando assim retrabalhos por falta de comunicação.

						\item Quão rápido sua equipe responde a mudanças;

							Metodologias ágeis, sofrerem muitas mudanças em seus requisitos, necessitam uma equipe que tenha capacidade de reagir rapidamente a alterações de escopo.

					\end{enumerate}
				\item \textbf{Processo};
					\begin{enumerate}
						\item Quanta documentção seu cliente exige;

							Metodologias tradicionais tendem a ter mais documentação que em metodologias ágeis, dependendo do projeto pode ser tanto uma vantagem como uma desvantagem.

						\item Quão frequentes devem ser as entregas do seu projeto;

							Em projetos tradicionais as entregas são mais espaçadas do que em projetos ágeis considerando a menor quantidade de reuniões disponíveis com o cliente.

					\end{enumerate}
				\item \textbf{Negócio}.
					\begin{enumerate}
						\item Quão crítico é seu projeto;

							Projetos críticos requerem uma documentação formal com precisão, o que é algo mais simples de se alcaçar em metodologias tradicionais.

						\item Qual a probabilidade dos requisitos do projeto mudarem;
						
							Projetos com mudaças intensas nos requisitos, geralmente tendem a ser projetos ágeis.

						\item 
					\end{enumerate}
			\end{enumerate}
		\item Ator responde questionário;
		\item Sistema calcula estisticamente qual rota deve ser utilizada, de acorodo com as respostas do ator;
			\label{item:1.3.1.1_empate}
		\item Sistema apresenta ao ator a escolha da metodologia;
		\item Usuário aceita a metodologia;
			\label{item:1.3.1.1_recusado}
		\item Sistema prepara a ferramenta para utilização da metodologia escolhida.
			\label{item:1.3.1.1_retorno1}
	\end{enumerate}

\paragraph{Fluxo alternativo A}

	\begin{enumerate}
		\item No passo \ref{item:1.3.1.1_empate} do fluxo básico, caso haja um empate entre metodologias;
		\item Sistema apresenta aos ao ator as metodologias empatadas e suas características;
		\item Ator escolhe a metodologia que deseja;
		\item O fluxo retorna para o passo \ref{item:1.3.1.1_retorno1} do fluxo básico.
	\end{enumerate}

\paragraph{Fluxo alternativo B}

	\begin{itemize}
		\item No passo \ref{item:1.3.1.1_recusado} do fluxo básico, caso o ator não aceite a metodologia proposta pelo sistema;
		\item Ator rejeita a opção da metodologia escolhida pelo sistema;
		\item Sistema apresenta todas as opções de metodologias cadastradas para que o usuário possa escolher;
		\item retorna para o passo \ref{item:1.3.1.1_retorno1} do fluxo básico.
	\end{itemize}	
	
\subsubsection{Caso de Uso - UC1.4.1.1 - Definir ``hibridez'' do projeto}
\paragraph{Descrição}
Este caso de uso auxilia a descobrir quão hibrido é o projeto proposto pelo cliente quais as caracteristicas ágeis e quais caracteristícas tradicionais que irão compor a "hibridez" do projeto.

\begin{enumerate}
	\item Atores
		Engenheiro de requisitos. 
	\item Pré-condições
		Não existe pré condições para este caso de uso.
	\item Pós-condições
		ao final do caso de uso deverão ser mostrados quais caracteristicas das abordagens o projeto irá herdar a fim de ver quão hibrido é o projeto" 
\end{enumerate}
