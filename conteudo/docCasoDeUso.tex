% !TEX root = ../main.tex

O documento de casos de uso tem como finalidade o detalhamento a fundo dos recursos do programa, aqui chamados de caso de uso, listados na sessão \ref{subsub:recursos_produto} deste documento, colocando todas as suas características, restrições e caminhos possíveis.

\subsection{Identificação dos atores}

Neste contexto, atores são representações genéricas de usuários do sistema, podende ser qualquer utilizador, sem se preocupar com nome do executor, apenas com sua função, esses atores estão listados na tabela \ref{tab:atores}.

\begin{table}[H]
\centering
\begin{tabular}{|l|p{8cm}|}

\hline
\textbf{Atores} &
\textbf{Descrição}
\\ \hline
%----------------------------------------------
Engenheiro de Requisitos &
Responsável, em metodologias tradicionais, pela elicitação dos requisitos e pela manutenção da rastreabilidade do sistema
\\ \hline
%----------------------------------------------
Analista de Requisitos &
Responsável, em metodologias tradicionais, gerência dos requisitos
\\ \hline
%----------------------------------------------
\textit{Product Owner} &
Responsável, em metodologias ágeis, por escrever histórias de usuário
\\ \hline
%----------------------------------------------
Equipe de portfólio &
Responsável, em metodologias ágeis, por gerir a parte do portfólio, como temas de investimento e épicos
\\ \hline
%----------------------------------------------
Equipe de programa &
Responsável, em metodologias ágeis, por gerir as features do sistema e manter a entrega das releases em dia
\\ \hline
%----------------------------------------------
Time &
Responsável, em metodologias ágeis, por implementar as histórias de usuário
\\ \hline
%----------------------------------------------
Cliente &
Responsável por validar os requisitos
\\ \hline

\end{tabular}
\caption{Atores do sistema}
\label{tab:atores}
\end{table}

\subsection{Diagrama de casos de uso}

\subsection{Detalhamento dos casos de uso}

%inserir modelo

\subsubsection{Caso de Uso: UC-01x Buscar cruzeiros}
