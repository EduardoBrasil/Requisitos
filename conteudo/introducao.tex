
\subsection{Contexto} % (fold)
\label{sub:contexto}

O desenvolvimento de software perpassa inúmeras fases até que o mesmo seja concluído e entregue ao cliente, uma dessas fases, e provavelmente a mais importante é a fase em que devemos entender o problema do usuário, compreender exatamente o que o mesmo necessita e apresentá-lo uma solução. Nesta fase, negociações serão feitas, tanto sobre funcionalidades do sistema quanto custo do projeto, tempo para conclusão do mesmo e restrições de projeto.
	
O resultado desta fase é uma documentação robusta, principalmente ao utilizar metodologias tradicionais de desenvolvimento. Nesta documentação se encontram as funcionalidades do software, as características do mesmo e as restrições de projeto, podendo abranger todo o software ou apenas uma primeira etapa de desenvolvimento, como é feito em metodologias ágeis, estes ítens são conhecidos como requisitos.

Com esta documentação em mãos, os desenvolvedores podem começar o desenvolvimento do sistema, porém muitos problemas surgem tanto na construção da documentação quanto na utilização da mesma para o desenvolvimento do software. A gerência, organização, classificação e rastreabilidade dos requisitos geram inúmeros problemas, já que o conteúdo é muito extenso para ser organizado facilmente de forma adequada. A partir deste problema, surge a necessidade da utilização de ferramentas que auxiliem na organização, classificação e rastreabilidade dos requisitos.

Inúmeras ferramentas já foram criadas para esta finalidade, contudo as ferramentas gratuitas não possuem a qualidade necessária para desenvolvimento de grandes e críticos sistemas. A partir daí, surge a necessidade da criação de uma nova ferramenta, de código aberto, grats e com qualidade suficiente para suprir as necessidades de todos os desenvolvedores de software do mundo.

%\subsection{Formulação do problema} % (fold)
%\label{sub:formula_o_do_problema}
%
%A disciplina de Manutenção e Evolução de Software possui a proposta de fazer com
%que os alunos do curso de Engenharia de Software contribuam com projetos reais
%que sejam software livre.
%Assim, os alunos tem a possibilidade de aplicar os conhecimentos adquiridos nas
%diversas disciplinas cursadas com contribuições em forma de código.
%
%O GestorPsi é um dos softwares utilizados na disciplina. Este não possui nem testes
%automatizados, nem funcionais, o que o torna uma excelente base de aplicação de
%técnicas de programação, também, o sistema não segue alguns \textit{code style guide}
%da comunidade Python, como PEP 008 e o PEP 020.
%
%A forma utilizada na disciplina para contribuição é através de \textit{fork}, ou seja,
%através da obtenção de uma cópia do código-fonte da aplicação no \textit{forge}
%(plataforma que abriga repositórios) que abriga o projeto, e após uma série de
%refatorações e implementações de código, \textit{pull requests} com as alterações
%feitas são submetidas para avaliação à equipe que mantém o sistema. Isso faz com
%que haja diferenças entre o repositório oficial do sistema e o que está sendo
%refatorado pelos alunos da Universidade de Brasília.

\subsection{Objetivos} % (fold)
\label{sub:objetivos}
\subsubsection{Objetivos Gerais}

[APRESENTAR OS OBJETIVOS GERAIS DO TRABALHO].

\subsubsection{Objetivos Específicos}
\begin{itemize}
	\item Entender e registrar o problema do cliente.
	\item Entender e registrar as necessidades do cliente.
	\item Definir requisitos funcionais para o desenvolvimento da ferramenta.
	\item Definir requisitos não-funcionais para o desenvolvimento da ferramenta.
	\item Definir restrições de projeto.
	\item Definir escopo e custo do projeto.
\end{itemize}
% [ Responder à questão: O QUE FAZER?
% Apresentar os objetivos como Objetivo Geral e Objetivos específicos.
% Objetivo Geral: o que se pretende atingir / alcançar com a pesquisa
% Objetivos específicos: São etapas do trabalho para se alcançar / atingir o objetivo geral. Utilizar verbos no infinitivo para descrever tais objetivos específicos.
% \begin{itemize}
% 	\item Exploratórios (conhecer, identificar, levantar, descobrir)
% 	\item Descritivos (caracterizar, descrever, traçar, determinar, definir)
% 	\item Explicativos (analisar, avaliar, verificar, explicar, validar)   ]
% \end{itemize}


\subsection{Justificativas} % (fold)
\label{sub:justificativas}

[APRESENTAR AQUI A JUSTIFICATIVA DA ESCOLHA DO TEMA DA DISCIPLINA. PRA QUE ESTAMOS FAZENDO ESSE TRABALHO?]
