% !TEX root = ../main.tex

O desenvolvimento de \sw~ passa por inúmeras fases até que seja concluído e entregue ao cliente, uma delas, e provavelmente a mais importante, é a \er, onde se deve entender o problema do usuário, compreender suas necessidades e apresentá-lo a uma solução. Nesta fase, serão feitas negociações sobre funcionalidades do sistema, custos, tempo para conclusão e restrições de qualquer tipo.

O resultado desta fase é uma documentação robusta, principalmente ao utilizar metodologias tradicionais de desenvolvimento. Nesta documentação encontram-se as funcionalidades do \sw, suas características e restrições, podendo abrange-lo completamente ou apenas uma etapa de desenvolvimento como é feito em metodologias ágeis.

A tarefa de construir e manter a documentação necessária em um projeto de \sw~ possui diversos problemas relacionados a diversas áreas diferentes, como por exemplo a gerência, organização, classificação e rastreabilidade dos requisitos. Surge assim a necessidade da utilização de ferramentas que possam amenizar as dificuldades encontradas.

\subsection{Propósito}

Ao ler este documento, todos os \stakeholder~ deverão compreender todo o contexto de negócio, os objetivos e escopo do projeto, assim como, entender o problema que deverá ser resolvido, quais necessidades do cliente deverão ser analisadas e quais serão as funcionalidades do sistema.

\subsection{Escopo}

Este documento abrange o contexto do desenvolvimento de \sw~ voltado para a \er, desde a elicitação à gerência de requisitos, e tem como objetivo levar o entendimento do projeto a qualquer leitor, desde leigos até especialistas na área. Encontra-se neste documento, o problema de negócio do cliente, suas reais necessidades, características e funcionalidades do sistema que foram possíveis mapear.

Dessa forma, a partir deste documento, pode-se obter conhecimento total sobre o projeto de desenvolvimento da \nomeferramenta, desde a metodologia utilizada até a forma de implementação do sistema.

\subsection{Definições, acrônimos e abreviações}

Durante o processo de elicitação e gerenciamento de requisitos é necessário que todos os envolvidos possam se comunicar sem que existam falhas de entendimento, para isso, foi desenvolvido um sumário contendo nomes que serão utilizados no processo, assim como suas definições.

\begin{itemize}

	\item \stakeholder

		Todas as partes envolvidas no contexto do sistema, desde o cliente e seus funcionarios até a equipe de desenvolvimento do sistema. Todos os interessados na solução de \sw~ são considerados \stakeholder~ do sistema \cite{sommerville2003engenharia}.

	\item \textit{Requisitos} 

		Engloba tudo que o \sw~ deve possuir para solucionar o problema em questão, desde funcionalidades do sistema até características que o \sw~ deve possuir.

	\item \textit{Requisitos Funcionais}

		São chamados de requisitos funcionais todos aqueles que apresentam as funcionalidades do sistema, tendo o mínimo de abstração possível \cite{sommerville2003engenharia}.

	\item \textit{Requisitos não Funcionais}

		São chamados requisitos não funcionais todos aqueles que apresentam as características do sistema, incluindo compatibilidade, o tempo de resposta ou qualquer outra exigência que não inclua funcionalidades \cite{sommerville2003engenharia}.

	\item \textit{Engenharia de Requisitos}

		Engenharia de Requisitos é um conceito que engloba todo um contexto de desenvolvimento de \sw~ que envolve elicitação de requisitos, negociação, verificação e validação, e documentação e gerência de requisitos para o desenvolvimento de um sistema computacional. O uso da palavra \textit{Engenharia} garante que técnicas sistematicas serão utilizadas para que os requisitos sejam completos, corretos e consistentes \cite{de2004analise}. 

	\item \textit{Fishbone ou Diagrama de Ishikawa}

		Consiste em uma técnica utilizada para o reconhecimento do macro problema do cliente. A utilização desta técnica garante uma facilidade maior para entender onde a solução deve atuar.

	\item \textit{Framework do problema}

		Consiste em uma técnica para organizar e auxiliar o entendimento do problema e apresentar aos stakeholders afetados o impacto gerado para o cliente e uma possível solução bem sucedida. A utilização do framework garante maior facilidade no entendimento do contexto do cliente.

	\item \textit{Framework de Necessidades}

		Consiste em uma técnica para oganizar uma tabela identificando necessidade, problema, solução atual e solução proposta. A utilização do framework de necessidade garante um melhor entendimento da necessidade do cliente.
 
	\item \textit{WorkShop}

		 Workshop é uma técnica de elicitação de requisitos na qual os partipantes discutem um problema em comum enquanto são aplicadas técnicas que ajudam em uma melhor identificação das necessidades do cliente e a melhoraram o rendimento das reuniões.

	\item \textit{Brainstorming}

		 Brainstorming é uma técnica de elicitação de requisitos que consiste em uma dinâmica de grupo para recolher ideias a respeito de um determinado assunto.

	\item \textit{Casos de Uso}

		 Caso de uso define uma sequência de ações que produz um resultado de valor observável. Os casos de uso fornecem estrutura para expressar requisitos funcionais no contexto dos processos de negócio e de sistema.

	\item \textit{Sprint}

		Representa o espaço de tempo no qual deverão ser realizadas atividades previamente estabelecidas para a resolução de um problema \cite{beck2000extreme}.

	\item \textit{Release}

		São entregas de código funcional, as quais são feitas por etapa, entregando pequenas partes do \sw~ \cite{beck2000extreme}.

	\item \textit{Product Owner (PO)}

		É o responsável pela atividade de repassar o conhecimento de todo o contexto de negócio para a equipe de desenvolvimento. Muitas vezes, o PO pode ser o próprio cliente ou qualquer funcionário que tenha conhecimento do problema e faz o intermédio entre a equipe de desenvolvimento e o cliente. \cite{beck2000extreme}

	\item \textit{Product Backlog}

		Representa a produção do trabalho executado durante o desenvolvimento \cite{sanches2010aplicaccao}.

	\item \textit{Sprint Backlog}

		Representa o trabalho a ser desenvolvido durante uma \textit{sprint} com o objetivo de criar um produto apresentável para a equipe. O \textit{backlog} da \textit{sprint} deve ser produzido de forma incremental.

\end{itemize}