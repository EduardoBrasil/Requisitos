\subsection{Contexto} % (fold)
\label{sub:contexto}

O desenvolvimento de software perpassa inúmeras fases até que o mesmo seja concluído e entregue ao cliente, uma dessas fases, e provavelmente a mais importante é a fase em que devemos entender o problema do usuário, compreender exatamente o que o mesmo necessita e apresentá-lo uma solução. Nesta fase, negociações serão feitas, tanto sobre funcionalidades do sistema quanto custo do projeto, tempo para conclusão do mesmo e restrições de projeto.

O resultado desta fase é uma documentação robusta, principalmente ao utilizar metodologias tradicionais de desenvolvimento. Nesta documentação se encontram as funcionalidades do software, as características do mesmo e as restrições de projeto, podendo abranger todo o software ou apenas uma primeira etapa de desenvolvimento, como é feito em metodologias ágeis, estes ítens são conhecidos como requisitos.

Com esta documentação em mãos, os desenvolvedores podem começar o desenvolvimento do sistema, porém muitos problemas surgem tanto na construção da documentação quanto na utilização da mesma para o desenvolvimento do software. A gerência, organização, classificação e rastreabilidade dos requisitos geram inúmeros problemas, já que o conteúdo é muito extenso para ser organizado facilmente de forma adequada. A partir deste problema, surge a necessidade da utilização de ferramentas que auxiliem na organização, classificação e rastreabilidade dos requisitos.

\subsection{Objetivos} % (fold)
\label{sub:objetivos}
\subsubsection{Objetivos Gerais}

O objetivos gerais deste projeto é a fase de requisitos e o início da implementação de uma ferramenta de gerência de requisitos \opensource{} que auxilie o usuário durante todo o processo de requisitos, desde a escolha da metodologia utilizada ao controle da rastreabilidade dos requisitos.

\subsubsection{Objetivos Específicos}

Objetivos especificos são pequenos marcos que serão realizados durante o projeto, são eles:

\begin{itemize}
	\item Entender e registrar os problemas da \er.
	\item Entender e registrar as necessidades da \er.
	\item Entender as caractrísticas da \er.
	\item Definir requisitos funcionais para o desenvolvimento da ferramenta.
	\item Definir requisitos não-funcionais para o desenvolvimento da ferramenta.
	\item Definir casos de uso do desenvolvimento da ferramenta.
	\item Definir escopo da ferramenta.
\end{itemize}

\subsection{Justificativas} % (fold)
\label{sub:justificativas}

Inúmeras ferramentas já foram criadas para esta finalidade, contudo as ferramentas gratuitas não possuem a qualidade necessária para desenvolvimento de grandes e críticos sistemas. A partir daí, surge a necessidade da criação de uma nova ferramenta \opensource{} com qualidade suficiente para suprir as necessidades de todos os desenvolvedores de software do mundo.

Outra vantagem de uma ferramenta \opensource{} é a possibilidade de qualquer interessado em manter a ferramenta estar permitido a fazer o mesmo, o que facilita uma futura evolução do projeto, mantendo ele sempre capaz de se manter reais as necessidades do usuário.